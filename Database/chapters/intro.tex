% !TEX encoding = UTF-8
% !TEX TS-program = pdflatex
% !TEX root = ../nt.tex
% !TEX spellcheck = it-IT

%************************************************
\chapter{Introduzione}
\label{cap:intro}
%************************************************\\

\begin{itemize}

\item{\textbf{Prerequisiti}}

Buona conoscenza di linguaggi \textit{Object-Oriented} (almeno uno), tecniche e strumenti. Elementi di computer networks e tecnologie di Rete, Web;

\item{\textbf{Abilità acquisite}}

Lo studente sarà in grado di progettare e capire i modelli dei dati, creare e gestire database e progettare ed implementare applicazioni data-centric.

\end{itemize}

Lo scopo è fornire le basi circa le principali teorie sui database, tecniche e strumenti per \textbf{usare} i database e \textbf{progettare/implementare} database \textbf{applications}.

\textbf{Argomenti}:

\begin{itemize}

\item Database, database relazionali, NoSQL e NewSQL;
\item Sistemi di gestione dei database (DBMS);
\item Modello Relazionale ed Algebra Relazionale;
\item SQL: definizioni dei dati e loro manipolazioni;
\item Basi della Computer-Human Interaction e progettazione delle interfacce;
\item Aspetti architetturali: Clients, Servers, Peers, Dispositivi, IoT, ...
\item Principi di Data-Analytics;
\item Analisi multidimensionale e data-warehouse;

\end{itemize}