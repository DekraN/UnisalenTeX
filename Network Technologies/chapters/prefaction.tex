% !TEX encoding = UTF-8
% !TEX TS-program = pdflatex
% !TEX root = ../nt.tex
% !TEX spellcheck = it-IT

Nel frattempo.. permetto al lettore di godersi cotanta passione e motivazione verso l'Ingegneria..

%************************************************
\chapter{Prefazione all'edizione italiana}
\label{cap:prefaction}
%************************************************\\

Alcuni anni fa, subito dopo la laurea, avendo ricevuto il mandato di far nascere un filone di attività nel settore delle reti a commutazione di pacchetto presso quello che allora era l'Istituto di Elettronica e Telecomunicazioni del Politecnico di Torino, mi fu consigliato di cominciare a leggere un libro appena uscito, decantandone le qualità di profondità e chiarezza.
Il libro era scritto da quella che stava emergendo come la massima autorità del settore a livello internazionale: il Prof. Leonard Kleinrock del Computer Science Department della mitica University of California a Los Angeles (UCLA).
Due anni dopo, mi ritrovai in mano il libro e di fronte l'autore in persona, mentre ero ritornato studente molto lontano da casa.
Non è quindi senza una certa emozione che scrivo queste poche righe di prefazione a un libro che per me, come per moltissimi, è stato una delle pietre miliari della formazione tecnica e culturale. Ciò sia per la fama dell'autore, sia per la chiarezza dell'esposizione, che antepone sempre la spiegazione intuitiva dei fenomeni ad una rigorosa trattazione matematica dei modelli.
La traduzione italiana, giungendo con qualche anno di ritardo rispetto all'edizione originale, ha giustamente tralasciato quelle parti del secondo volume che hanno maggiormente sofferto per la rapidissima evoluzione di un settore in continuo divenire. Essa raccoglie in un unico testo tutta la trattazione teorica, la cui importanza continua a crescere con il passare del tempo, trovando sempre nuovi settori applicativi.
Si può auspicare che la disponibilità del testo in italiano serva a diffondere sempre di più l'insegnamento della teoria delle code, rendendo così possibile un approccio quantitativo nell'analisi degli aspetti tipici delle telecomunicazioni, dell'informatica e dei trasporti.
Da parte mia un augurio ai lettori: che possano trovare in questo libro il gusto per lo studio quantitativo dei fenomeni, l'interpretazione dei risultati e la conseguente comprensione del comportamento dei sistemi. Quest'ultima è, in fondo, l'essenza stessa della ricerca scientifica.

\bigskip

\begin{flushright}
MARCO AJMONE MARSAN\\
Professore di Reti di Telecomunicazioni\\al Politecnico di Torino\\
Settembre 1991
\end{flushright}