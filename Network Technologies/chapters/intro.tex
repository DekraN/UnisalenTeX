% !TEX encoding = UTF-8
% !TEX TS-program = pdflatex
% !TEX root = ../nt.tex
% !TEX spellcheck = it-IT

%************************************************
\chapter{Introduzione}
\label{cap:intro}
%************************************************\\

L'argomento principale del corso è la Progettazione di Reti ad Alta Disponibilità. Vengono coperti gli aspetti fondamentali, ovvero i concetti di \textbf{Affidabilità} e \textbf{Disponibilità} e loro interrelazioni. Saranno studiate le performance delle reti, supporti alle comunicazioni wireless, verranno fornite le basi sulla sicurezza, e sarà fatta una importante review sullo Stato dell'Arte relativo a queste problematiche. Le performance e la sicurezza sono dei requisiti che spesso entrano in conflitto tra di loro. Si pensi ad esempio a quanto possa essere differente una realtà universitaria da una realtà bancaria relativamente a questi requisiti. Nella realtà bancaria un downtime annuale accettabile è di circa 5 minuti. Si parlerà di disponibilità 5-9. Requisiti principali: \{\textbf{\underline{affidabilità}, \underline{disponibilità} e \underline{performance}}\}. Requisiti QoS (Quality of Service). \textit{Reliability} and \textit{Availability}.
Un esempio veramente forte ed indicativo del livello tecnologico delle moderne Reti può essere dato dalle comunicazioni veicolari, ove si sfrutta il protocollo \textit{IEEE 802.11p}, che sarebbe informalmente il Wi-Fi esteso al mondo veicolare.. In questi casi vi è un forte spirito collaborativo in queste particolari reti, chiamate Reti Ad-Hoc.

Negli ultimi anni le reti giocano un ruolo molto importante sulle aziende. Sono motivo di profitto e produttività. Queste reti, molto critiche, devono soddisfare alcuni requisiti (prestazioni, QoS). Performance, mobilità, affidabilità e sicurezza sono i tipici requirements. Internet è stata messa su servizio QoS. Le reti possono essere molto diverse: si pensi ad esempio al servizio VoIP il quale utilizza Internet per trasmettere il contenuto video.

Noi ci riferiamo sempre alle IP Networks. Reliability = Affidabilità, QoS. Un campione audio deve arrivare entro 2 ms. Requisito invece stringente sulla probabilità d'errore per servizi Web, FTP, etc. Queste ultime si basano tutte sul TCP. Invece in applicazioni Real-Time si utilizza l'UDP: bisogna fare in fretta. Una rete QoS deve soddisfare vari requisiti. Di recente si parla anche di \textit{Ethernet deterministica} (che riguarda la QoS). Banda per comunicazioni in tempo reale. BUS-CAN per comunicazioni sulle varie centraline delle auto, per il mondo veicolare. Si vogliono quindi, in alcune situazioni, privilegiare alcuni flussi di comunicazioni rispetto ad altri. \{\textbf{Mobilità}, \textbf{Affidabilità}, \textbf{Disponibilità}, \textbf{Sicurezza}\}. Questi sono i requisiti tipici QoS. Bisogna comunque sempre fare i conti con i costi. \underline{Reduce costs}! A tal proposito si fornisce un cenno su spese \textit{OPEX} e \textit{CAPEX}. Nel nostro ambito, rispettivamente le prime riguardano le spese per l'operatività della rete, mentre le seconde riguardano spese per la manutenzione e mantenimento della rete.

Una rete ad Alta Disponibilità necessita di \underline{RIDONDANZA}. I requisiti sulla sicurezza spesso possono entrare in conflitto con i requisiti di velocità. Bisogna scegliere di volta in volta il miglior compromesso.
Problemi odierni: Sono cruciali le applicazioni di rete per le aziende. Fondamentale è in questo senso il supporto alla QoS. Stanno aumentando il numero delle applicazioni che si basano su paradigma \textit{BYOD-BYOA}, rispettivamente \textit{Bring-Your-Own-Device} e \textit{Bring-Your-Own-Access}. Sempre più aziende stanno permettendo di far lavorare il dipendente con il proprio dispositivo. \`E naturale pensare a problemi in termini di sicurezza. Il supporto alla QoS è quindi fondamentale.

Problemi relativi all'utilizzo delle WAN (reti geografiche). Ci sarà una WAN che interconnette queste reti aziendali, con un traffico 80/20, ovvero 80\% traffico locale e 20\% traffico che attraversa la WAN. Vecchie topologie \textit{Hub-And-Spoke} (a stella). Servono invece diversi percorsi alternativi. Abbiamo bisogno di reti parzialmente magliate. Adesso la situazione è cambiata: il rapporto 80/20 di cui sopra è stato invertito! Le reti locali sono molto affidabili. Su una WAN non è così: ha un \textit{Packet Loss Rate} abbastanza elevato. Oggi l'80\% del traffico attraversa la WAN. Le aziende adesso stanno raggruppando i server in Data Center. Bisogna attraversare la WAN quindi per interagire con i server che ora stanno in questi Data Center. Si fa un cenno a tal proposito ai cosiddetti \textit{Chatty Protocols}.
Le aziende stanno comunque riducendo il numero dei Data Center. Per effettuare una transazione c'è bisogno di un'applicazione. Un unico data center diverrebbe presto uno SPoF (\textit{Single Point of Failure}). In questo caso bisognerebbe attraversare la WAN molte volte. All'utente interessa proprio il tempo di risposta! Questo è quindi il prezzo da pagare quando si attraversa una WAN: i tempi di risposta sono molto elevati. Ma non tutti i protocolli sono di tipo chatty. Prendiamo l'HTTP. Oggi le applicazioni sono distribuite. Anche se il protocollo non è chatty (HTTP), comunque si richiede di scaricare molti oggetti oggigiorno. Il prezzo temporale è legato all'attraversamento della WAN. Abbiamo quindi dei problemi di prestazioni. Si parla molto oggi inoltre della virtualizzazione dei server. Corrispondono a delle vere e proprie macchine virtuali. Il traffico diventa però non più predicibile, e si richiede che la banda venga quindi dimensionata correttamente.

Dal momento che questi flussi di traffico non sono più predicibili, per far fronte ai problemi di prestazioni, l'industria del networking ha implementato delle ottimizzazioni. Ma in che modo si abbassano le prestazioni? Ci sono dei problemi con le WAN, legati a come si comporta il TCP. Quindi si creano delle caratteristiche favorevoli che consentono di migliorare le prestazioni delle reti. Infatti in queste ottimizzazioni possiamo trovare un aumento di velocità 40x per l'HTTP.

Cloud Computing. Nuvole che mettono a disposizione dei servizi. Abbiamo vari paradigmi a tal proposito: \{\textit{Software-as-a-Service} (\textbf{SaaS}), \textit{Platform-as-a-Service} (\textbf{PaaS}), \textit{Infrastructure-as-a-Service} (\textbf{IaaS})\}. Apparati di rete virtualizzati e messi in rete. Tutto si basa sul concetto di virtualizzazione. Provider di servizi che mettono a disposizione questi servizi. Noi ne usufruiamo mediante cloud. Non è affatto sorprendente che anche qui abbiamo vari problemi di performance, oltre che di sicurezza. \`E importante approfondire i concetti di PoP (\textit{Points of Presence}). I PoP dovrebbero quindi essere interconnessi mediante reti high-performance ed highly secure.

Abbiamo quindi capito che è molto importante prestare particolare attenzione a QoS, disponibilità, affidabilità e sicurezza.
5-9 99.999\%. \`E una sigla, un numero che ha a che fare con l'Alta Disponibilità (High-Availability). Si ha in questo caso un downtime annuo di 5.26 minuti, ovvero 25.9 secondi per mese, quindi circa 6 secondi a settimana. Per ottenere questo risultato c'è ovviamente bisogno di RIDONDANZA. Il discorso sicurezza è molto legato alla disponibilità. La sicurezza è un fattore molto importante di cui tener conto quando si trattano reti ad alta disponibilità. \`E molto importante quindi valutare la \textit{Capacità di Resilienza} di un dato sistema, ovvero la sua capacità di recuperare in seguito a guasti od in generale situazioni sfavorevoli. Si tenga presente che Availability $\neq$ Reliability. L'Affidabilità $R(t)$ è la probabilità che il tempo di vita di un dispositivo sia maggiore di una certa quantità temporale $\tau$. Invece la disponibilità (di un servizio in generale) è la frazione di tempo in cui un servizio è UP. (es 5-9, 99.999\%). La sigla prima citata significa che per il 99.999\% di tempo il servizio dev'essere UP.

\textit{Software Defined Networking} (SDN) - NFD. Tutto questo mette in crisi le architetture tradizionali di networking. Oggi i dispositivi all'interno delle reti sono veramente tanti (sempre per far fronte a sicurezza, QoS, etc.). Serve una massiccia riconfigurazione degli apparati. E se i flussi di traffico cambiano, bisogna andare a configurare altri apparati. Alla fine non impongono queste policy. Il SDN consente di soddisfare questi requisiti in maniera programmabile. Separa il piano di controllo (Intelligenza) dal piano di forwarding (piano dati). L'intelligenza è centralizzata in un controllo SDN. Poi vi sono apparati SDN-aware che hanno adesso solo la capacità di inoltrare i dati. Non hanno più intelligenza ma solo capacità di forwarding. Poi abbiamo ovviamente apparati che gestiscono l'intelligenza. Si è standardizzato il protocollo mediante il quale l'SDN può alla fine controllare i vari dispositivi (\textit{OpenFlow}). In base ad esso vengono identificati i flussi dati. Gli apparati OpenFlow enabled ricevono informazioni dal nodo centrale di intelligenza (Brain).

\textit{Open-Networking-Foundation} (ONF) sta sviluppando l'SDN. Poi abbiamo il paradigma NFV (\textit{Network Functions Virtualization}). Tutto quello che accade in Europa lo gestisce l'ETSI. E l'ETSI sta dietro l'NFV, la quale è una tecnologia che virtualizza i dispositivi di rete. La realizzazione in SW dei dispositivi di rete NFV ed SDN rapresenta la ricerca attule del mondo delle reti. La Rete viene ormai vista come un vero e proprio dispositivo virtuale programmabile.
Dispositivi \textit{IEEE 802.11p}. Standard per le Comunicazioni Veicolari. Architettura EITS. Comitato europeo che standardizza tutte le comunicazioni.
