% !TEX encoding = UTF-8
% !TEX TS-program = pdflatex
% !TEX root = ../eng.tex
% !TEX spellcheck = it-IT

%************************************************
\chapter{Cambridge English for Engineering - Esercizi Risolti}
\label{cap:cmbeng}
%************************************************\\

\section{UNIT 1: Technology in Use}

\subsection{2}

\subsubsection{a}

\textbf{Paula, a design engineer for a CPS manufacturer, is discussing product development with José, a senior manager new to the company. Listen to the conversation and complete the following notes}:

\begin{itemize}

\item the primary application of GPS: \textit{navigation};
\item associated applications:

\begin{itemize}
\item Tracking systems for: \textit{(monitoring) delivery vehicle};
\item Tracking systems for: \textit{(finding) stolen cars}.
\end{itemize}

\item more creative features:

\begin{itemize}
\item \textit{drift alarms};
\item \textit{man overboard button}.
\end{itemize}

\item not technical innovations: \textit{innovative use of} technology.

\end{itemize}

\subsubsection{b}

\textbf{Complete the following extracts from the discussion with words that come from use}:

\begin{itemize}
\item Then you've got associated applications, \textit{uses} that are related to navigating;
\item .. tracking system you can \textit{use} for monitoring delivery vehicles;
\item .. from the end-\textit{user's} point of view, accuracy is no longer the main selling point. Most devices are accurate enough. The key is to make them more useful.
\end{itemize}

\subsection{3}

\subsubsection{a}

\textbf{Match the GPS applications (1-6) to the descriptions (a-f)}:

\begin{itemize}

\item \textit{topographical surveying}: mapping surface features;
\item \textit{geological exploration}: applications in mining and the oil industry;
\item \textit{civil engineering}: setting out positions and levels of new structures;
\item \textit{avionics equipment}: air traffic control, navigation and autopilot systems;
\item \textit{maritime applications}: navigation and safety at sea;
\item \textit{GPS in cars and trucks}: highway navigation and vehicle tracking.

\end{itemize}

\subsection{4}

\subsubsection{a}

\textbf{Complete the following extracts from the conversation by underlining the correct words}

\begin{itemize}
\item there's a setting on the GPS that \textit{allows} it to detect the movement;
\item an alarm sounds to warn you, and \textit{prevents} the boat from drifting unnoticed;
\item and \textit{ensures} that you don't lose track of where you were, which then \textit{enables} you to turn round and come back to the same point..
\end{itemize}

\subsubsection{b}

\textbf{Match the words in Exercise 4a to the synonyms}:

\begin{itemize}

\item \textit{ensures}: make sure;
\item \textit{enables/allows}: permits;
\item \textit{prevents}: stops.

\end{itemize}

\subsubsection{c}

\textbf{Complete the following extract from the user's manual of a GPS device using the verbs in Exercise 4a. Sometimes, more than one answer is possible}:

The core function of your GPS receiver is to \textit{allow/enable} you to locate your precise geographical position. To \textit{allow/enable} - the device to function, it receives at least three signals simultaneously from the GPS constellation - 30 dedicated satellites which \textit{ensure} - receivers can function anywhere on earth. To \textit{allow/enable} - extremely precise positioning and \textit{prevent} errors from occurring due to external factors, this device is designed to receive four separate signals (see enhanced system accuracy on page 18).

\subsection{6}

\subsubsection{c}

\textbf{Match the verbs (l -9) from the text in Exercise 6b to the definitions (a-i)}:

\begin{itemize}

\item \textit{connecting}: joining;
\item \textit{raise}: lift / make something go up;
\item \textit{transported}: carried (objects, over a distance);
\item \textit{support}: hold something firmly / bear its weight;
\item \textit{attached}: fixed;
\item \textit{ascend}: climb up;
\item \textit{descend}: climb down;
\item \textit{powered}: provided with energy / moved by a force;
\item \textit{controlled}: driven / have movement directed.

\end{itemize}

\subsection{7}

\subsubsection{a}

\textbf{James, an engineer, is giving a talk on space elevators. Complete his notes using the correct form of the verbs (l-7) in Exercise 6c}:

\begin{itemize}

\item Challenge of \textit{connecting} a satellite to earth by cable is significant;
\item To \textit{support} its own weight, and be securely \textit{attached} at each end, cable would need phenomenal strength-to-weight ratio;
\item How could vehicles be \textit{raised} into space, up cable?
\item Self-contained energy source problematic, due to weight (heavy fuel or batteries required to \textit{power} vehicle);
\item Two possible ways round problem:

\begin{itemize}

\item Transmit electricity wirelessly. But technique only at research stage;
\item Solar power. But would only allow vehicle to \textit{ascend} slowly. Not necessarily a problem, as car could be controlled remotely, allowing it to \textit{transport} payloads unmanned.

\end{itemize}
\end{itemize}

\subsection{12}

\subsubsection{a}

\textbf{Complete the following tips on emphasising technical advantages using the words in the box}:

When describing technical advantages, it's useful to emphasise ... 

\begin{itemize}

\item \textit{enhanced} performance, compared with the older model of the same product;
\item negative issues that have been \textit{reduced}, or completely \textit{eliminated};
\item special features that differentiate the technology from \textit{conventional} systems;
\item performance levels that make the technology \textit{superior} to the competition.

\end{itemize}

\subsubsection{c}

\textbf{Complete the following sentences from the briefing by underllning the correct emphasising word}:

\begin{itemize}

\item We've come up with a \underline{\textbf{completely}} unique profile;
\item It \underline{\textbf{dramatically}} reduces vibration;
\item Machines like this can never be \underline{\textbf{entirely}} free from vibration;
\item The new design runs \underline{\textbf{extremely}} smoothly;
\item Another advantage of the new profile is that it's \underline{\textbf{considerably}} lighter;
\item So compared with our previous range, it's \underline{\textbf{highly}} efficient;
\item Trials so far suggest the design is \underline{\textbf{exceptionally}} durable;
\item We expect it to be \underline{\textbf{significantly}} more reliable than rival units.

\end{itemize}

\subsubsection{d}

\textbf{Match the words in Exercise 12c to the synonyms}:

\begin{itemize}

\item \textit{entirely/totally}: completely;
\item \textit{considerably/dramatically}: significantly;
\item \textit{exceptionally/highly}: extremely.

\end{itemize}

\subsection{13}

\textbf{You are Otis engineers back in the l850s, when elevators were new. ln pairs, prepare a short talk to brief your sales colleagues on the advantages of elevators for lifting people and goods. Emphasise the points below using the phrases and techniques from this section. Remember that people at this time are sceptical about the technology}:

\begin{lstlisting}[breaklines=true]
- Have you ever tried to get a full shop bag for five or more stories of stairs? It is not so easy, I would say.
- An elevator will sort things up! It's a comfortable way to simplify your life. No more shop bags or water bottles, for example.
- It will considerably lower your work. And I know what you will think. It's unsafe! It's Dangerous! WRONG!
- It's actually safer than climbing stairs. As you can see in this blackboard, the car is attached with a steel cable that can support several tons. 
- Surely safer than climbing stairs and maybe undergo a fall due to a slippery surface. And it's not so difficult to use as you can think.
- Not more than climbing stairs!
- The lift is controlled from the inside of the car, simply pressing a button. Just by doing this, you can climb from the basement up to the top of a skyscraper. 
- And don't forget about the economic aspects of the things. A significantly large use of lifts will help the economy of our great America. 
- How, shall you ask?
- Using lifts we can build taller buildings. And no one will complain about living at the top floor. This way, the value of the land will extremely skyrocket! 

\end{lstlisting}

\begin{flushright}Marco Chiarelli\\Matteo Settembrini\\03/11/2015\end{flushright}


\section{UNIT 2: Materials Technology}

\subsection{2}

\subsubsection{a}

\textbf{Read the following web page and complete the missing headings using the words in the box}:

\begin{itemize}

\item \textit{Steel} Scrap can be sorted easily using magnetism. If the metal is galvanised (coated with zinc) the zinc is fully recyclable. If it is stainless steel, other metals mixed with the iron, such as chromium and nickel, can also be recovered and recycled;

\item \textit{Glass} Sorting is critical, as there are key difference between the clear and coloured material used in bottles and jars, and the high-grade material used in engineering applications, which contains traces of metals;

\item \textit{Copper} Scarcity makes recycling especially desirable, and justifies the cost of removing insulation from electric wires, which are a major source of scrap. Pure metal can also be recovered from alloys derived from it, notably brass (which also contains quantities of zinc and often lead) and bronze (which contains tin);

\item \textit{Aluminium} The cost of melting down existing metal is significantly cheaper than the energy-intensive process of electrolysis, which is required to extract new metal from ore;

\item \textit{Timber} Hardwood and softwood can be reused. However, the frequent need to remove ironmongery and saw or plane off damaged edges, can make the process costly;

\item \textit{Rubber} Tyres are the primary source of recyclable material. These can be reused whole in certain applications. They can also be ground into crumbs which have varied uses;

\item \textit{Plastic} An obstacle to recycling is the need to sort waste carefully. While some types can be melted down for reuse, many cannot, or result in low-grade material.

\end{itemize}

\subsubsection{b}

\textbf{Match the materials from the web page (l-8) in Exercise 2 to the definitions (a-h)}:

\begin{itemize}

\item \textit{stainless steel}: a type of steel not needing a protective coating, as it doesn't rust;

\item \textit{zinc}: a metal used to make brass, and in galvanised coatings on steel;

\item \textit{iron}: the predominant metal in steel;

\item \textit{bronze}: an alloy made from copper and tin;

\item \textit{lead}: a dense, poisonous metal;

\item \textit{hardwood}: timber from deciduous trees;

\item \textit{ore}: rocks from which metals can be extracted.

\item \textit{softwood}: timber from pine trees.

\end{itemize}

\subsubsection{c}

\textbf{Complete the following sentences using from, with or of}:

\begin{itemize}

\item Bronze contains significant amounts \textit{of} copper;
\item Galvanised steel is steel coated \textit{with} zinc;
\item Steel is an alloy derived \textit{from} iron;
\item Pure metals can usually be recovered \textit{from} alloys;
\item To produce stainless steel, iron is mixed \textit{with} other materials;
\item Stainless steel contains quantities \textit{of} chromium and nickels;
\item Glass tableware contains traces \textit{of} metals, such as lead;
\item When new metal is extracted \textit{from} ore, the costs can be high.

\end{itemize}

\subsection{3}

\subsubsection{b}

\textbf{Listen to an extract from the talk and compare your ideas with what Irina says. What example does she use to illustrate her main point?}:

The main point that Irina makes is that it's important to consider the total environmental impact of a product, including producing it (preuse), using it (in-use) and recycling it (post-use). She gives the example of an energy-saving light bulb.

\subsubsection{c}

\textbf{Irina asks the engineers to do a simplified environmental audit. Their task is to compare steel and aluminium car bodywork from an ecological perspective. Listen to Sophia and Pete, two of the engineers, discussing the topic and make notes of their ideas}:

Sophia and Pete's ideas: Pre-use: aluminium production (extraction from ore and recycling), coating steel (galvanising), transporting and handling bulk material, cutting and welding In-use: weight (impact on fuel consumption), lifespan (frequency of manufacturing).

\subsection{5}

\subsubsection{a}

\textbf{Read the article on braking systems. In the title of the article, what do the colours green and red refer to?}

Green refers to ecological issues. Red refers to heat (red hot means very hot). Also, a hot topic is a current important topic.

\subsubsection{b}

\textbf{In pairs, answer the following questions}:

\begin{itemize}

\item Why do most braking systems waste energy? \textit{Because they use friction, which wastes energy as heat};

\item What are regenerative braking systems, and how do they save energy? \textit{They recover heat and use it to power the car};

\item What characteristics are required of materials used for the brakes on racing cars? \textit{The ability to generate high levels of friction, and to resist the effects of friction and consequent heat};

\item What is meant by heat soak, and why is it a problem in racing cars? \textit{Heat from the engine being absorbed by the chassis, which can damage sensitive parts such as electronic components and plastic parts}.

\end{itemize}

\subsubsection{c}

\textbf{Match the materials from the text (l-7) to the descriptions (a-g)}:

\begin{itemize}

\item \textit{compounds}: combinations of materials;
\item \textit{exotic}: rare or complex;
\item \textit{ferrous}: iron and steel;
\item \textit{ceramics}: minerals transformed by heat;
\item \textit{alloy}: mixture of metals;
\item \textit{non-metallic}: materials that are not metal;
\item \textit{polymers}: plastic materials.

\end{itemize}

\subsection{6}

\subsubsection{e}

\textbf{Match the parts of the cable (a-e) in Exercise 6c to the following categories of materials (l -5). You will need to use some parts more than once}:

\begin{itemize}

\item non-metallic: \textit{insulation, waterproof membrane, outer jacket};
\item metallic: \textit{armoured protection, conductor};
\item ferrous metal: \textit{armoured protection};
\item non-ferrous metal: \textit{conductor};
\item polymer-based: \textit{insulation, waterproof membrane, outer jacket}.

\end{itemize}

\subsection{8}

\subsubsection{c}

\textbf{Find words in the text in Exercise 8b to match the following definitions}:

\begin{itemize}

\item \textit{toughness} = the opposite of fragility;
\item \textit{abrasion resistance} = resistance to damage caused by friction;
\item \textit{thermal stability} = resistance to problems caused by temperature change;
\item \textit{durable} = long lasting;
\item \textit{lightweight} = the opposite of heavy;

\end{itemize}

\subsection{9}

\subsubsection{a}

\textbf{Match the automotive parts (l -5) to the descriptions (a-e)}:

\begin{itemize}

\item \textit{drive belts}: flexible bands used in transmission systems;
\item \textit{brake pads}: pads pressed against discs to induce deceleration;
\item \textit{tyres}: pneumatic envelopes in contact with the road surface;
\item \textit{sealing gaskets}: sheets inserted between parts to prevent gas or fluid leakage;
\item \textit{bullet-resistant armour}: protective barriers capable of resisting gunshot.

\end{itemize}

\textbf{Read the information from DuPont'u on the following page explaining some of the automotive applications of Kevlar@. Complete the text using the automotive parts in Exercise 9a}:

\begin{itemize}

\item Car and truck \textit{tyres} have incorporated Kevlar into their construction because it offers superb puncture, abrasion and tear resistance;

\item The high modulus and abrasion resistance of Kevlar help \textit{drive belts} retain their original shape and tension over the millions of revolutions they go through over the lifespan of a vehicle;

\item The frictional forces that \textit{brake pads} are designed to endure take less of a toll on those made with Kevlar pulp. The enhanced thermal stability and inherent abrasion resistance of Kevlar allow them to last long and stop the vehicle safely and quietly;

\item Kevlar provides an effective, lightweight \textit{bullet-resistant armour} solution for vehicles that require protection against ballistic attack, allowing cars and light trucks to retain most of their original handling characteristics;

\item Chemical stability and thermal stability help make \textit{sealing gaskets} reinforced with Kevlar pulp strong and durable. The galvanic corrosion resistance of Kevlar also contribute to long-term engine performance.

\end{itemize}

\subsection{10}

\subsubsection{a}

\textbf{Listen to a conversation about the properties of materials used in a specific type of tool and answer the following questions}:

\begin{itemize}

\item Where does the conversation take place? \textit{At the dentist's};
\item What tool is being discussed? \textit{The tool is a dental drill};
\item Which materials can be used for its different parts? \textit{Titanium can be used for the handle, and tungsten-carbide and diamond for the bur}.

\end{itemize}

\subsubsection{b}

\textbf{Complete the following extracts from the conversation using the properties in Exercise 8c. Listen again and check your answers}:

\begin{itemize}

\item The handle mustn't be heavy. Ideally, you want it to be \textit{lightweight};
\item Resisting friction is essential. The key requirement is \textit{abrasion resistance};
\item The bur has to be built to last. Obviously, they need to be very \textit{durable};
\item Heat builds up in the bur. You need a good degree of \textit{thermal stability}.

\end{itemize}

\subsubsection{c}

\textbf{Match the words and phrases (1-5) from Exercise 10b to the synonyms (a-e)}:

\begin{itemize}

\item \textit{ideally}: for the best results;
\item \textit{obviously}: it's clear that;
\item \textit{the last thing you want}: the worst situation;
\item \textit{the key requirement}: the most important factor;
\item \textit{a good degree of}: a lot of / a high level of.

\end{itemize}

\subsection{14}

\subsubsection{c}

\textbf{Listen to the following phrases from the conversation and underline the stressed syllable. Practise saying the phrases}:

\begin{itemize}

\item not par\underline{tic}ularly suitable;
\item ex\underline{cep}tionally resistant;
\item not at \underline{all} suitable; 
\item tre\underline{men}dously marketable;
\item \underline{rel}atively complex;
\item \underline{not} at all that good.

\end{itemize}

\subsubsection{d}

\textbf{Complete the following table using the words in the box}:

\begin{itemize}

\item extremely, exceptionally, tremendously;
\item quite, fairly, pretty, relatively;
\item not very, not particularly, not (all) that;
\item not enough, insufficiently, not adequately;
\item definitely not, not at all.

\end{itemize} 

\section{UNIT 3: Components and assemblies}

\subsection{2}

\subsubsection{c}

\textbf{Erin, an engineer with the same company, is describing different electrical plug and socket formats during the briefing. Listen and match the descriptions (1-6) to the pictures (a-f)}:

\begin{itemize}

\item a: \textit{triangular};
\item b: \textit{cylindrical};
\item c: \textit{circular};
\item d: \textit{linear};
\item e: \textit{rounded};
\item f: \textit{rectangular};

\end{itemize}

\subsubsection{d}

\textbf{Complete the following phrases from the descriptions using adjectives based on the words in brackets}:

\begin{itemize}

\item there are \textit{circular} pins for live and neutral. (circle);
\item  the earth slot's got a flat base with one side \textit{rounded} over to form a semi-circle (round);
\item This one has \textit{rectangular} blades for live, neutral and earth... (rectangle);
\item it has a \textit{cylindrical} slot to receive the earth pin (cylinder);
\item the pins are arranged in \textit{linear} configuration (line);
\item they're laid out in \textit{triangular} configuration (triangle).

\end{itemize}

\subsubsection{e}

\textbf{Listen and underline the stressed syllable in each of the following words}:

\begin{itemize}

\item \underline{rec}tangle;
\item rec\underline{tan}gular;
\item \underline{tri}angle;
\item tri\underline{ang}ular;
\item \underline{cy}linder;
\item cy\underline{lin}drical;
\item line;
\item \underline{lin}ear.

\end{itemize}

\subsection{3}

\subsubsection{a}

\textbf{Listen to a longer description from the meeting. Which picture (a-f) in Exercise 2c does Erin describe?}

\textit{cylindrical} 

\subsubsection{b}

\textbf{Complete the following extracts from the description using the correct form of the words in the box}:

\begin{itemize}

\item  there's a circular slot at the top. It's obviously a blind \textit{hole}, it doesn't go right through;
\item there are two plastic \textit{ridges}, one of either side of the plug casing, and they slot into corresponding \textit{grooves} at each side of the socket. In addition, the centre of the socket is \textit{recessed}. So rather than being \textit{flush with} the front of the socket, on the same face, the circular area that receives the plug is \textit{set back} from the surrounding casing;
\item These covers only open when pressure is applied to both by the two \textit{pin} of the plug simultaneously.

\end{itemize}

\subsection{4}

\subsubsection{a}

\textbf{Andy and Karin, two electrical engineers, are evaluating a plug and socket format in Exercise 2c. Listen to the conversation and make notes of the advantages and disadvantages of the following features}:

\begin{itemize}

\item Plug slots into a recess in the socket: 
\begin{itemize}
\item Advantages:
\begin{itemize}
\item\textit{The plug resists pullout forces};
\item\textit{Nothing can touch the pins if the plug is partially pulled out};
\end{itemize}
\item Disadvantages: \textit{It's difficult to pull out};
\end{itemize}
\item Covers protect live and neutral slots:
\begin{itemize}
\item Advantages: \textit{Children can't stick things in the socket};
\item Disadvantages: \textit{If the mechanism is too sensitive, it can be difficult to insert the plug}
\end{itemize}

\end{itemize}

\subsection{6}

\subsubsection{a}

\textbf{Complete the following training material for graduate engineers using the words in the box}:

MANUFACTURING TECHNIQUE EVALUATION: CUTTING OPERATIONS 

Key factors in determining the most appropriate cutting technique are: material characteristics (notably hardness, and thermal and electrical properties), component thickness, component shape and complexity, required edge quality, and production volume. Select cutting options below for a detailed analysis of techniques.

CUTTING OPTIONS:

\begin{itemize}

\item\textit{Sawing}: abrasive cutting, removing a kerf of material. Includes cutting with toothed blades and abrasive wheels;
\item\textit{Shearing}: use of pressure on smooth-edged blades for guillotining and punching;
\item\textit{Drilling}: removal of materials across the full diameter of a hole, or using hole-saws for cutting circumferential kerfs;
\item\textit{Milling}: removal of surface layers with multiple cutting wheel passes;
\item\textit{Flame-cutting}: using oxy-fuel (oxygen + combustible gas, often acetylene).

\end{itemize}

\subsubsection{c}

\textbf{Complete the following definitions using the words in the box}:

\begin{itemize}

\item A \textit{punch} makes holes by applying pressure to shear the material;
\item A \textit{guillotine} makes straight cuts by applying pressure to shear the material;
\item A \textit{kerf} is the width of the saw cut;
\item A \textit{toothed blade} has sharp edges for cutting or milling;
\item An \textit{abrasive wheel} has a hard, rough surface for cutting or grinding;
\item A \textit{hole-saw} cuts a circular piece to remove an intact core of material.

\end{itemize}

\subsection{7}

\subsubsection{a}

\textbf{Read the following extract of promotional literature from a leading producer of ultra-high-pressure (UHP) waterjet cutting machines. ln pairs, explain the phrases in bold}

\begin{itemize}

\item\textbf{secondary operations}: additional machining, such as polishing;
\item\textbf{net-shaped parts}: parts with accurately cut edges; often intricate shapes;
\item\textbf{heat-affected zone}: the area modified by high temperatures (resulting from the heat of cutting);
\item\textbf{mechanical stresses}: physical forces such as shear forces when sawing or guillotining metal;
\item\textbf{narrow kerf}: narrow thickness of material removed during cutting: especially easy to do with waterjet cutting;
\item\textbf{tightly nested}: when several components are cut from the same piece of material the components can be placed close together, making better use of the material.

\end{itemize}

\subsubsection{b}

\textbf{Evan is talking to Mr Barrett about UHP waterjet cutting. Listen to the conversation and match the phrases in the box to the extracts (1-4)}:

\begin{itemize}

\item Extract 1: \textit{net-shaped parts};
\item Extract 2: \textit{heat-affected zone};
\item Extract 3: \textit{mechanical stresses};
\item Extract 4: \textit{narrow kerf}.

\end{itemize}

\subsubsection{c}

\textbf{Complete the following extracts from the conversation by underlining the correct phrases}:

\begin{itemize}

\item So they are \textit{especially good when} gou have intricate shapes;
\item Saw blades are obviously \textit{useless} when you're cutting curved shapes;
\item sawing is \textit{not the best solution} if you want to avoid altering the material;
\item it's \textit{ideal for} for metals. 

\end{itemize}

\subsection{10}

\subsubsection{b}

\textbf{Complete the following table using the words in the box}:

\begin{itemize}

\item{\textbf{Mechanical fixings}}:
\begin{itemize}
\item\textit{bolt};
\item\textit{screw};
\item\textit{rivet};
\item\textit{clip}.
\end{itemize}
\item{\textbf{Non-mechanical fixings}}:
\begin{itemize}
\item\textit{weld};
\item\textit{adhesive}.
\end{itemize}

\end{itemize}

\subsubsection{c}

\textbf{Label the photos (l-5) with the words in Exercise l0b}:

\begin{itemize}

\item\textit{weld};
\item\textit{bolt};
\item\textit{adhesive};
\item\textit{screw};
\item\textit{rivet};
\item\textit{clip}.

\end{itemize}

\subsubsection{d}

\begin{itemize}

\item \{\textit{connecting, joining, fixing}\}: describes any kind of connection;
\item \{\textit{bolting, riveting}\}: describes mechanical connections only;
\item \{\textit{bonding, gluing, welding}\}: describes non-mechanical connections.

\end{itemize}

\subsection{11}

\subsubsection{a}

\textbf{Complete the following questions using the words in the box}:

\begin{itemize}

\item How can we fix these two components \textit{together}?
\item How can we fix these two components to \textit{each other}?
\item How can we fix these two components \textit{on}?
\item How can we fix these two components \textit{to / onto} this component?

\end{itemize}

\subsubsection{b}

\textbf{Complete the following training web page using the words in Exercise 11a}:

The most suitable method of joining components depends on many factors, which extend beyond the obvious issue of required strength.

\begin{itemize}

\item Will the joint need to be disconnected in the future? If a part is bolted \textit{on}, it can obviously be removed at a later date. If two components are bonded to \textit{each other} with strong adhesive, or welded \textit{together}, then subsequent removal will clearly be more difficult;

\item What external factors might affect the joint? Water or heat can weaken adhesive joints. And no matter how tightly nuts are screwed \textit{onto} bolts, vibration can cause them to work loose over time;

\item How quality-sensitive is the jointing technique? Components are rarely joined \textit{to} each other in ideal conditions. Inadequately tightened fixings, improperly prepared surfaces, or flawed welds are inevitable. How could such imperfections affect the joint negatively? 

\end{itemize}

\subsection{13}

\subsubsection{c}

\textbf{Answer the questions in Exercise 13b}:

\begin{itemize}

\item How did the actual flight differ from the one that was planned? \textit{The balloons climbed faster than expected, then entered controlled airspace adjacent to an airport};

\item What incidents occurred just before and just after the landing? \textit{A rope tangled with a power line, then Mr Walters was arrested};

\item What is said about the modern equivalent of this type of activity? \textit{The modern equivalent, cluster ballooning, is not a mainstream sport, but is becoming more popular};

\item What components were used to assemble the flying machine? \textit{A garden chair, helium filled Weather balloons and ropes}.

\end{itemize}

\subsection{14}

\subsubsection{a}

\begin{itemize}

\item\textit{above, over};
\item\textit{below, beneath, underneath};
\item\textit{alongside, adjacent to, beside};
\item\textit{around (attorno)};
\item\textit{outside};
\item\textit{inside, within}.

\end{itemize}

\subsubsection{b}

\textbf{Complete the following sentences about the flying garden chair using the prepositions in the box. Check your answers against the text in Exercise 13b}:

\begin{itemize}

\item Projecting \textit{above} the chair was a cluster of ropes, tied to 42 helium-filled weather balloons;
\item Anchor ropes were fastened \textit{around} the bumper of his cars;
\item Larry Walters had an airgun inserted \textit{in} his pocket,
\item The helium contained \textit{within} the balloons warmed up in the sun;
\item After takeoff, the anchor ropes remained suspended \textit{beneath} the chair.

\end{itemize}

\subsubsection{c}

\textbf{Complete the following descriptions of how the garden chair airship was assembled by underlining the correct words}:

\begin{itemize}

\item A quantity of helium gas was \textit{contained} inside each balloon;
\item A tube was \textit{inserted} inside the openings of the balloons, to inflate them;
\item The balloons were \textit{situated} over the chair, in a large cluster;
\item The chair was \textit{suspended} under the balloons by ropes;
\item Arm rests, \textit{located} beside the pilot, at each side, helped to hold him in place;
\item The landing gear, \textit{projecting} below the seat, consisted, simply, of the chair legs;
\item The pilot was \textit{positioned} underneath the balloons, so his weight was low down.

\end{itemize}

\subsubsection{d}

\textbf{Which two other words have the same meaning as positioned?}

\begin{itemize}

\item\textit{located}
\item\textit{situated}

\end{itemize}

\section{UNIT 4: Engineering design}

\subsection{2}

\subsubsection{b}

\textbf{Complete the following definitions using the types of drawing in the box}:

\begin{itemize}

\item A \textit{plan} gives a view of the whole deck, from above;
\item An \textit{elevation} gives a view of all the panels, from the front;
\item An \textit{exploded view} gives a deconstructed view of how the panels are fixed together;
\item A \textit{cross-section} gives a cutaway view of the joint between two panels;
\item A \textit{schematic} gives a simplified representation of a network of air ducts;
\item A \textit{note} gives a brief description or a reference to another related drawing;
\item A \textit{specification} gives detailed written technical descriptions of the panels.

\end{itemize}

\subsection{3}

\subsubsection{c}

\textbf{Complete the following extracts from the conversation and explain what is meant by each one}:

\begin{itemize}

\item Is this drawing \textit{to} scale? \textit{Do the dimensions correspond with a scale?}
\item It's one \textit{to} five. \textit{The dimensions on the drawing are 1/5 of their real size};
\item You shouldn't scale \textit{off} drawings. \textit{You shouldn't measure dimensions on a drawing using a scale rule and take them to be exact};
\item it's actual size, on a \textit{full-scale} drawing; \textit{ The dimensions on the drawing are the same as their real size}.

\end{itemize}

\subsection{6}

\subsubsection{b}

\textbf{Read the technical advice web page and answer the following questions}:

\begin{itemize}

\item{\textbf{How is a superflat floor different from an ordinary concrete floor?}} \textit{A superflat floor has a much flatter surface. It's finished more precisely than an ordinary concrete floor};
\item{\textbf{What accuracy can be achieved with ordinary slabs, and with superflat slabs?}} \textit{Ordinary slabs can be flat to $\pm 5mm$. Superflat slabs can be flat to within $1mm$};
\item{\textbf{What problem is described in high bay warehouses}} \textit{Slight variations in floor level can cause forklifts to tilt, causing the forks to hit racks or drop items}.

\end{itemize}

\subsubsection{d}

\textbf{Complete the following expressions from the web page which are used to describe tolerances}:

\begin{itemize}

\item \textit{within} tolerancee (inside the limits of a given tolerance);
\item \textit{plus} or \textit{minus} $5mm\ (\pm 5mm)$;
\item \textit{tight} tolerance (close tolerance);
\item \textit{outside} tolerance (not inside the limits of tolerance).

\end{itemize}

\subsubsection{e}

\textbf{Complete the following sentences using the expressions in Exercise 6d}:

\begin{itemize}

\item The frame's too big for the opening. The opening's the right size, so the frame must be \textit{outside tolerance};
\item The total tolerance is $1mm$. The permissible variation either side of the ideal is $\pm 0.5mm$;
\item The engineer specified $\pm 5mm$ for the slab finish, and we got it to $\pm 2mm$. So it's well \textit{within tolerance};
\item You can't finish concrete to $\pm 0.1mm$. There's no way you can work to such \textit{tight tolerance};

\end{itemize}

\subsection{7}

\subsubsection{b}

\textbf{Complete the following table using the words in the text in Exercise 6b and audioscript 4.3 on page 89}:

\begin{itemize}

\item

\begin{itemize}
\item What's the \textit{length}?
\item Is it \textit{long}?
\item Is it \textit{short}?
\end{itemize}

\item

\begin{itemize}
\item What's the \textit{width}?
\item Is it \textit{wide}?
\item Is it \textit{narrow}?
\end{itemize}

\item

\begin{itemize}
\item What's the \textit{depth}?
\item Is it \textit{high}?
\item Is it \textit{low}?
\end{itemize}

\item

\begin{itemize}
\item What's the \textit{thickness}? 
\item Is it \textit{thick}?
\item Is it \textit{thin}?
\end{itemize}

\item

\begin{itemize}
\item What's the \textit{height}?
\item Is it \textit{deep}?
\item Is it \textit{shallow}?
\end{itemize}

\end{itemize}

\subsubsection{c}

\textbf{Mei has done a revised drawing for the floor slab. Read the extract from her email about the new design and complete the message using the correct form of the words in Exercise 7b}:

Please find attached a revised drawing for the floor slab, now reconfigured for defined movement. In order to accommodate guided vehicle $1080mm$ \textit{wide} (as specified by the client) we propose a standard \textit{width} of $1280mm$ for each superflat lane. At $14.5m$, the \textit{length} of the longest lane on the network is within the maximum slab run that can be cast in a single concrete pour, thus avoiding construction joints on straight runs. On curved sections, a standard $8.5m$ turning radius is used, as per the guided vehicle manufacturer's recommendations.
In order to allow for the eventuality of future grinding, we have located the top layer of reinforcement $10mm$ deeper below the slab surface. This additional \textit{depth} has not, however, been added to the overall slab \textit{thickness}, which remains $275mm$. The reinforcing bars also remain in $12mm$ diameter. As a result, the levels of wall-mounted process installations - many of which need to be fixed at a precise \textit{height} above finished floor level - are unaffected.

\subsubsection{d}

\textbf{Which two words in the email relate to circles? What aspects of a circle do they describe?}

\begin{itemize}

\item\textit{diameter}: the maximum width of a circle;
\item\textit{radius}: the distance from the centre of a circle to its circumference (half the diameter).

\end{itemize}

\subsection{10}

\subsubsection{a}

\textbf{The following extracts from emails relate to a project to build an indoor ski complex in Australia, using artificial snow. The messages were circulated by an engineer to members of the design team, and to a specialist contractor. Read the emails and, in pairs, answer the following questions. Note that the emails are not in the correct order}:

\begin{itemize}

\item What are all the emails about? \textit{design information (at different stages of the design process)};
\item What different types of documents are mentioned?
 \textit{sketches, design brief, revised/amended drawing, superseded drawing, preliminary drawing, working drawing, summary/notes}.
 
\end{itemize}

\subsubsection{b}

\textbf{Put the emails in the correct sequence}:

\begin{itemize}

\item b
\item d
\item c
\item a
\item e

\end{itemize}

\subsubsection{c}

\textbf{Complete the following definitions using the types of drawing in the box}:

\begin{itemize}

\item A \textit{sketch} is a rough drawing of initial ideas, also used when production problems require engineers to amend design details and issue them to the workfore immediately; 
\item A \textit{design brief} is a written summary intended to specify design objectives;
\item A \textit{working drawing} is an approved drawing used for manufacturing or installation. There is often a need to revise these drawings to resolve production problems. In this case, amended versions are issued to supercede the previous ones;
\item A \textit{preliminary drawing} is a detailed drawing that colleagues and consultans are invited to approve if they accept them, or comment on if they wish to request any changes.

\end{itemize}

\subsection{13}

\subsubsection{a}

\textbf{The following records are from the indoor ski complex project. They show correspondence between the design team and construction team. Read through the texts quickly and answer the following questions}:

\begin{itemize}

\item What is the general subject of the correspondence? \textit{Design problems and solutions};
\item What is meant by query and instruction? \textit{A query is a question. An instruction is an explanation of what to do / official permission to do something};
\item Some queries refer to earlier conversations. Suggest why these have been followed up in writing. \textit{Written follow-up is important in order to keep a record for contractual/financial purposes};
\item What is meant by \textit{dwg} and \textit{dims}? \textit{drawing} and \textit{dimensions}.

\end{itemize}

\subsubsection{c}

\textbf{Complete the following pairs of sentences using the verbs in the box}:

\begin{itemize}

\item The components are in each other's way. = The components \textit{clash};
\item Please ask for more information. = Please \textit{request} more information;
\item Can I suggest a solution to the problem? = Can I \textit{propose} a solution?
\item Please instruct the supplier to send the parts to this address. = Please \textit{advise} the supplier;
\item  Any conflicting details must be queried. = You must \textit{clarify} any conflicting details;

\end{itemize}

\subsection{14}

\subsubsection{b}

\textbf{Chen, a technician, is explaining the problem in Exercise 14ato Ron, an engineer. Complete the conversation using the words in the box}:

\begin{itemize}

\item{\textbf{Chen}}: There's a \textit{discrepancy} between these details that you might be able to \textit{clarify} straight away. On the plan of this plate, it shows eight bolts. But on section A, here, there are no bolts shown in the middle. So there would be only six, which obviously \textit{contradicts} the plan. But as you can see, this plate's going to be bolted on a T profile. So we couldn't put a row of bolts down the middle, because they'd \textit{clash} with the flange running along the middle of the T. So I'd \textit{propose} just going for two rows of bolts. The \textit{alternative} would be to redesign the T section, which would obviously be a bigger job.

\item{\textbf{Ron}}: Yes. Let's go to for two row of bolts, \textit{as per} the sections.
\item{\textbf{Chen}}: OK, fine. Will you send an email to \textit{confirm} that?

\end{itemize}


\section{UNIT 6: Technical development}

\subsection{2}

\subsubsection{b}

\textbf{How do Claudia and Kevin focus on specific subjects? Complete the following phrases from the conversation using the words in the box. Listen again and check your answers}:

\begin{itemize}

\item with \textit{regard} to the capacity;
\item in \textit{terms} of the number of people;
\item as far as size is \textit{concerned};
\item and as \textit{regards} the graphics;
\item \textit{regarding} the schedule.

\end{itemize}

\subsubsection{c}

\textbf{Write questions using the following prompts and the phrases in Exercise 2b}:

\begin{itemize}

\item dimensions: what / overall size / module? \textit{With regard to the dimensions, what is the overall size of the module?}
\item materials: what / bodywork / made of? \textit{Concerning the materials, what is the bodywork made of?}
\item schedule: when / work start? \textit{Regarding the schedule, when will the work start?}
\item power: what / maximum output / need / be? \textit{In terms of power}, what will the maximum output need to be?
\item heat resistance: what sort / temperature / paint / need / withstand? \textit{As regards heat resistance, what sort of temperature will the paint need to withstand?}
\item tolerance: what level / precision / you want us / work to? \textit{Concerning tolerance, what level of precision do you want us to work to?}

\end{itemize}

\subsection{3}

\subsubsection{b}

\textbf{Listen again and explain what is meant by the words and phrases in bold}:

\begin{itemize}

\item \textbf{to what extent} do you want the experience to be physical? \textit{how much};
\item \textbf{The degree to which} it moves can be varied.. \textit{the amount};
\item it's obviously difficult to \textbf{quantify} something like this ... \textit{calculate / give a quantity};
\item The only way to \textbf{determine} what's right is to actually sit in a simulator ... \textit{judge / decide};
\item you can \textbf{assess} the possibilities. \textit{measure / test}.

\end{itemize}

\subsubsection{c}

\textbf{Following the meeting, Claudia writes an email to update Rod, an engineering colleague. Read the extract and choose a word or phrase from Exercise 3b that means the same as the words in bold. Sometimes more than one answer is possible}:

In order to (\textit{assess}) \textbf{find out about} the simulator's dynamic capabilities, we looked at the types of effect the simulator should produce, and (\textit{the degree to which}) \textbf{the amount} these physical effects should be felt by passengers. Specifically, the following issues were discussed:

\begin{itemize}

\item (\textit{to what extent}) \textbf{How severely} should the module generate vibration, to simulate engine thrust?
\item How much buffeting should be simulated? That is, (\textit{the degree to which}) \textbf{how severely} the module generates jolting, due to supposed atmospheric turbulence;
\item (\textit{to what extent}) \textbf{How much} will passengers be exposed to constant linear G-force, to simulate deceleration?

\end{itemize}

In order to (\textit{quantify}) \textbf{work out} the magnitude of the above parameters, it was decided that the prototype will be equipped with variable controls. This will enable the client to (\textit{assess}) \textbf{evaluate} different levels of severity through trials inside the simulator.

\subsection{6}

\subsubsection{a}

\textbf{Read the newspaper article and answer the following questions}:

\begin{itemize}

\item How is the statue being made, and what is it being made from? \textit{It's being carved from a block of sandstone};
\item What is Rick Gilliam's role? \textit{He's overseeing the logistics of the project};
\item What will the statue be placed on in its final position in front of the museum? \textit{On a stone plinth};
\item What technical problem did they have to solve? \textit{How to stop the slings from getting trapped beneath the statue, so they can be withdrawn, after the statue has been lowered onto the plinth by crane};

\end{itemize}

\subsubsection{d}

\textbf{ Complete the following suggestions from the conversation using the words in the box}:

\begin{itemize}

\item Why \textit{not} come up with a way of hooking onto the side of the statue?
\item Well, \textit{couldn't} we drill into it, horizontally?
\item We \textit{could} fill all the holes, couldn't we?
\item Or, \textit{alternatively}, we could make sure the holes were out of sight;
\item What \textit{about} drilling into the top, vertically?
\item I suppose \textit{another} option would be to use some sort of grab, on the end of the crane jib;
\item Why \textit{don't} we ask them?

\end{itemize}

\subsection{9}

\subsubsection{e}

\textbf{Complete the following expressions from the conversation using the words in the box and indicate the degree of feasibility each expression describes}:

\begin{itemize}

\item It'll be \textit{dead} easy;
\item It'll cost \textit{peanuts};
\item It'll be quite a \textit{painstaking} job;
\item It's \textit{perfectly} feasible;
\item It's achievable, but it's \textit{stretching} it;
\item there's no \textit{way} you can do it;
\item It's \textit{borderline};
\item It's a \textit{tall} order;
\item It'll take \textit{forever};
\item It'll cost an arm and a \textit{leg}.

\end{itemize}

\subsection{12}

\subsubsection{d}

\textbf{Look at the following verbs from the discussion and find three examples where re- means again. Match the other three verbs to the definitions in the box}:

\begin{itemize}

\item redesign: \textit{design again};
\item reinvent: \textit{invent again};
\item refine: \textit{improve the details};
\item revamp: \textit{improve overall};
\item rethink: \textit{think again};
\item remain: \textit{stay (the same)}.

\end{itemize}

\subsubsection{e}

\textbf{Complete the following expressions from the discussion using the words in the box. Listen and check your answers}:

\begin{itemize}

\item \textit{reinvent} the \textit{wheel};
\item designing the whole thing from the \textit{ground up};
\item \textit{room} for \textit{improvement};
\item \textit{Achilles heel};
\item \textit{back} to the \textit{drawing board};
\item make a \textit{quantum leap};
\item designing the system from \textit{scratch}.

\end{itemize}

\subsubsection{f}

\textbf{Match the expressions (l -5) in Exercise l2e to the definitions (a-0)}:

\begin{itemize}

\item waste time re-creating something that has already been created: \textit{reinvent the wheel};
\item the biggest weakness: \textit{Achilles heel};
\item start again because the first plan failed: \textit{back to the drawing board};
\item make huge progress: \textit{make a quantum leap};
\item design from the beginning: \textit{designing the system from scratch / designing the whole thing from the ground up};
\item potential for doing a better job: \textit{room for improvement}.

\end{itemize}

\subsubsection{g}

\textbf{Rewrite the following sentences using the correct form of the expressions in Exercise l2}:

\begin{itemize}

\item Unfortunately, we had to scrap the concept and start again: \textit{We had to go back to the drawing board};
\item This problem is the product's most serious shortcoming: \textit{This problem is the product's Achilles heel};
\item There's no point redesigning what already works perfectly well: \textit{There's no point reinventing the wheel};
\item It's a totally new design - we started from the very beginning: \textit{It's a totally new design - We started from the ground up};
\item The new design is so much better - it's a transformation: \textit{The new design is a quantum leap};
\item I think there's definitely a possibility to do better in this area: \textit{I think there's room for improvement in this area}.

\end{itemize}


\section{UNIT 8: Monitoring and control}

\subsection{2}

\subsubsection{c}

\textbf{Match the words in the box to the synonyms (l -5)}:

\begin{itemize}

\item sensor / \textit{detector};
\item measurement / \textit{reading};
\item control (adjust) / \textit{regulate};
\item sense / \textit{detect} / \textit{pick up};
\item activate / \textit{set off} / \textit{trigger}.

\end{itemize}

\subsubsection{d}

\textbf{Complete the following extracts from the conversation by underlining the correct words}:

\begin{itemize}

\item Not just the usual system that \textit{activate} the lights;
\item We could use presence detectors to \textit{control} other systems;
\item a presence detector \textit{senses} that everyone's left a meeting room;
\item a temperature sensor picks up a positive \textit{reading};
\item the sensor \textit{detects} sunlight, and \textit{triggers} the blinds;
\item those sensors \textit{set off} a circulation system;
\item we'd use presence detectors and heat sensors to \textit{regulate} as many systems as possible?

\end{itemize}

\subsection{5}

\subsubsection{a}

\textbf{Match the sensor or measuring system (l -5) to the industrial applications (a-e)}:

\begin{itemize}

\item pressure measurement: \textit{checking the force exerted by steam inside a vessel};
\item temperature measurement: \textit{ measuring the level of heat generated by an exothermic reaction};
\item flow measurement: \textit{monitoring the speed of water travelling around a supply pipe};
\item level measurement: \textit{monitoring the amount of ethanol contained in a storage tank};
\item process recorders: \textit{monitoring the number of cans moving along a conveyor belt}.

\end{itemize}

\subsection{6}

\subsubsection{b} 

\textbf{Match the words (1-10) from the discussion to the definitions (a-j)}:

\begin{itemize}

\item input: \textit{the entry value, for example at the start of a process};
\item output: \textit{the exit value, for example at the end of a process};
\item optimum: \textit{the best / the most effective/efficient};
\item differential: \textit{the gap between two values};
\item consumption: \textit{the amount of supplies/fuel used};
\item cumulative: \textit{the total quantity so far};
\item rate: \textit{a value often expressed with per, for example units per hour};
\item cycle: \textit{all the steps in a process, from start to finish};
\item frequency: \textit{how often something happens};
\item timescale: \textit{a specified period}.

\end{itemize}

\subsubsection{c}

\textbf{The following specification was written following the conversation. Complete the text using the words in Exercise 5b}:

\underline{Vessel Bl: Sensor and Measuring System Requirements}

Two pressure sensors: one located inside the vessel, and a second situated on the pipe running downstream, to enable any pressure \textit{differential} to be detected.
A flow meter to monitor gas \textit{consumption}. Data will be recorded as a \textit{cumulative} figure (total usage), and as flow \textit{rate}, in litres per second. Note: Software will be configured to log flow against the \textit{timescale} of a system clock, in order to pinpoint peak flow periods occurring between the start and the finish of a given reaction \textit{cycle} and to assess the \textit{frequency} with which they occurr.

Two temperature sensors: one at the entry point of the vessel, to measure \textit{input} temperature, and a second at the outlet point to monitor \textit{output} temperature. Note: Precise regulation of the entry temperature will be key to obtaining \textit{optimum} reaction performance.

\subsection{8}

\subsubsection{e}

\textbf{Match the words (1-8) from the talk to the definitions (a-h)}:

\begin{itemize}

\item continuous: \textit{without interruption};
\item fluctuations: \textit{changes, movements in general};
\item peaks and troughs: \textit{high points and low points on a graph curve};
\item peak demand: \textit{maximum power requirement at a given time};
\item range: \textit{amount between an upper and lower limit};
\item band of fluctuation: \textit{zone of up-and-down movement};
\item blips: \textit{momentary rises followed by a fall};
\item continual: \textit{regular and repetitive}.

\end{itemize}

\subsection{9}

\subsubsection{a}

\textbf{Read the document on energy saving aimed at industrial plant and facility managers. Complete the text using the words in Exercise 8e}:

Dynamic demand control systems can be fitted to electrical appliances that operate on duty cycles, i.e. appliances that start up, run for a time, shut down again, and then remain on standby for a while before repeating the same cycle. Heating and refrigeration units are common examples of power-hungry equipment that operate on this start-run-stop-wait basis.
Dynamic systems exploit the fact that duty cycle appliances do not require \textit{continuous} power. The purpose of the systems is to help smooth power demand for the benefit of electric utilities. To achieve this, they delay the start-up of the appliances they control during periods of \textit{peak demand}. However, only minor adjustments are made to timing as, generally, the appliances concerned can only be held on standby for short periods as they need to run on a \textit{continual} basis. But this still benefits electric utilities as it helps to avoid problematic, momentary \textit{blips} on the demand curve.
Dynamic controls work by detecting slight \textit{fluctuations} in the frequency of the mains AC supply. Although this varies only within a very narrow \textit{range}, small drops in frequency indicate that power station turbines are working close to full capacity. The dynamic control system can therefore hold the appliance on standby for a short time until mains frequency increases again.

\subsection{11}

\subsubsection{a}

\textbf{Read the email extract and answer the following questions}:

\begin{itemize}

\item Who do you think sent the email? What is their role within the company? \textit{A senior manager};
\item What type of review is the company going to undertake? \textit{A review of the company's organisation and facilities};
\item  What is the obiective of the review? \textit{Optimising efficiency / the use of engineers' skills};

\end{itemize}

\subsubsection{d}

\textbf{Complete the following sentences using the words or phrases in the box. Sometimes more than one answer is possible}:

\begin{itemize}

\item They asked for a \textit{ballpark figure} for setting up the new system;
\item I've got the figures in my computer, but I couldn't tell you \textit{off the top of my head};
\item The work is \textit{pretty much} finished, there's just the tidying up to do;
\item The actual cost of the stadium was \textit{nowhere near} the estimate at $£2m$ over budget;
\item I think it'll take \textit{somewhere in the region of} two weeks to complete the report;
\item The development will cost \textit{roughly} $\$10m$.

\end{itemize}

\subsection{12}

\subsubsection{b}

\textbf{Find words and phrases in audioscript 8.9 on page 93 to match the following definitions}:

\begin{itemize}

\item approximately: \textit{roughly};
\item much more than / \textit{well over};
\item at least / \textit{a good} (two thirds);
\item most / \textit{the vast majority};
\item almost zero / \textit{next to nothing}.

\end{itemize}

\subsubsection{c}

\textbf{Complete the following replies to express the figures in approximate terms using the words in Exercises 11d and l2b. Sometimes more than one answer is possible}:

\begin{itemize}

\item How old is this equipment? A \textit{good} five years old;
\item What percentage of the PCs need changing? \textit{Pretty much} all of them;
\item How many of the computers are up to spec? \textit{Nowhere near} all of them;
\item How many of the staff use the CAD system? \textit{Roughly} half of them;
\item How much would the new printers cost? \textit{Well over} $\$2000$;
\item How much does an adapter like this cost? \textit{Next to nothing};
\item How long would a full system take to install? \textit{Somewhere in the region of} 5 days;
\item Can most of our clients read these files? Yes \textit{the vast majority} of them.

\end{itemize}


\section{UNIT 9: Theory and practice}

\subsection{2}

\subsubsection{c}

\textbf{Listen again and complete the following extracts from the conversation using the words and phrases in the box}:

\begin{itemize}

\item the tests would obviously be \textit{virtual}, based on a computer model;
\item go into a wind tunnel, with a scale model, or a full-size \textit{mock-up};
\item it's not just about data gathering. You also have to \textit{validate} the data;
\item The \textit{acid test} only comes when you try out a full-scale prototype in real conditions. We need to make sure that everything is \textit{tried-and-tested} outside, with a full-scale \textit{trial run};
\item with changeable weather, it's not easy to do \textit{back to back testing} out \textit{in the field}.

\end{itemize}

\subsubsection{d}

\textbf{Match the words and phrases in Exercise 2c to the definitions (a-h)}:

\begin{itemize}

\item a 3D model simulating shape and size, but without internal components: \textit{mock-up};
\item proven to be reliable through real use / trial: \textit{tried and tested};
\item outdoors. in a real situation: \textit{in the field};
\item describes something simulated by software, not physical: \textit{virtual};
\item a crucial trial to prove whether or not something works: \textit{the acid test};
\item trials to compare two different solutions, in the same conditions: \textit{back to back testing};
\item prove theoretical concepts by testing them in reality: \textit{validate};
\item a practical test of something new or unknown to discover its effectiveness: \textit{trial run}.

\end{itemize}

\subsubsection{e}

\textbf{Complete the aerodynamic design development plan of the energy-efficient vehicle using stages (a-e)}:

\begin{itemize}

\item Experiment using CFD software;
\item \textit{Narrow down design options to three, based on computer data};
\item Produce reduced-scale mock-ups of designs and test in wind tunnel;
\item \textit{Select best design, based on data from wind tunnel tests};
\item Build first full-scale mock-up;
\item \textit{Test model in wind tunnel to validate data from scale tests};
\item Produce two revised designs to improve on full-scale mock-up;
\item \textit{Carry out back-to-back tests in wind tunnel with mock-up};
\item Select best design, based on data from tests;
\item \textit{Build full-size working prototype};
\item Carry out field tests with trial runs outside.

\end{itemize}


\subsection{5}

\subsubsection{a}

\textbf{Rephrase the words in brackets to complete the following extracts from the conversation}:

\begin{itemize}

\item So, \textit{theoretically}, the horizontal speed will keep decreasing;
\item So, \textit{assuming} the drop altitude's very low;
\item \textit{Surely}, a low vertical speed is the critical factor;
\item Because, \textit{presumably} if the groundspeed's quite high, there's a danger the container will roll;
\item So, \textit{arguably}, rolling is the worst problem.

\end{itemize}

\subsubsection{b}

\textbf{Rephrase the words in bold in the following sentences using the words in Exercise 5a}:

\begin{itemize}

\item \textit{Presumably} there'll always be a certain amount of groundspeed;
\item \textit{Assuming} the container will roll, we'll need to protect it accordingly;
\item \textit{Theoretically} groundspeed will almost always be positive;
\item \textit{Arguably} it's inevitable the container will roll and bounce along;
\item \textit{Surely} high vertical speed is less problematic than high groundspeed.

\end{itemize}

\subsubsection{c}

\textbf{In pairs, decide whether the following words and phrases are used to agree or disagree. can you think of other phrases for agreeing and disagreeing?}

\begin{itemize}

\item\textbf{Agree}:
\begin{itemize}
\item Sure;
\item Absolutely;
\item True;
\item Of course;
\item\textbf{\textit{Other phrases}}: I totally/completely agree
\end{itemize}

\item\textbf{Disagree}:
\begin{itemize}
\item I'm not so sure;
\item I'm not convinced;
\item Not necessarily;
\item\textbf{\textit{Other phrases}}:
\begin{itemize}
\item I'm not sure I agree;
\item I disagree;
\item I totally disagree;
\end{itemize}
\end{itemize}

\end{itemize}

\subsection{7}

\subsubsection{b}

\textbf{Manfred Haug, an aeronautical engineer, is describing his early rocket experiments. Read the description and explain what is meant by the expressions in bold}:

\begin{itemize}

\item\textbf{trial and error}: \textit{means testing ideas to see what happens. The expression implies that the testing process is not very scientific, and is simply based on guesswork};
\item\textbf{Unfamiliar territory}: \textit{means an unknown subject, an area where someone lacks experience};
\item\textbf{On a steep learning curve}: \textit{means learning rapidly, often as a result of being put in an unfamiliar situation without the necessary knowledge or experience}.

\end{itemize}

\subsection{8}

\subsubsection{b}

\textbf{Read the following extracts from the interview. What is meant by the words in bold?}

\begin{itemize}

\item we \textbf{expected} it would shoot up reasonably fast: \textit{though / predicted};
\item we \textbf{didn't anticipate} just how powerful it would be: \textit{didn't expect/predict};
\item \textbf{It totally exceeded our expectations}: \textit{it was much better than we had hoped}.

\end{itemize}

\subsection{9}

\subsubsection{c}

\textbf{Usten again and complete the following phrases from the description}:

\begin{itemize}

\item (as expected) It didn't go exactly \textit{according to plan};
\item (extremely well) It worked \textit{a treat}.

\end{itemize}

\subsection{10}

\subsubsection{c}

\textbf{Read the following phrases that Manfred uses. complete the definitions by underlining the correct words}:

\begin{itemize}

\item as it turned out: what happened in \textit{practice};
\item what actually happened: what happened in \textit{practice};
\item we underestimated the pressure: it was \textit{more} than we thought;
\item we overestimated the strength: it was \textit{less} than we thought;
\item plastic bottles ore hardly up to the job: they're \textit{inadequate};
\item I learned the hard way: it was a \textit{practical} lesson.

\end{itemize}

\subsection{13}

\subsubsection{b}

\textbf{Read the article and answer the following question}:

\begin{itemize}

\item What are chicken cannons designed to do? \textit{To fire dead chickens in order to test aircraft engines and windshields for their resistance to bird strikes};
\item Why was a chicken cannon used for a train test? \textit{Because it was a high-speed train and bird strikes were a potential danger};
\item What were the effects of the test? \textit{The chicken broke through both the windshield and the back of the driver's compartment}.

\end{itemize}

\subsubsection{c}

\textbf{The text in Exercise I 3b is an urban legend (or urban myth) - a commonly told story that is said to be true, but which is not. Can you guess what temperature issue caused the unexpected effects?}

\textit{They used a frozen chicken}.

\subsubsection{d}

\textbf{Complete the following sentences using the words and phrases in the box}:

\begin{itemize}

\item Bird strikes can \textit{result in} damage to aircraft;
\item Bird strikes were a potential problem for the train \textit{because of / due to / owing to} its speed;
\item During the test, the train was severely damaged as a \textit{result of} the impact;
\item The damage occurred \textit{because of} a problem relating to temperature;
\item The impact of the chicken \textit{caused} it to enter the train;
\item The engineers thought the gun was faulty so \textit{consequently} they called their colleagues.

\end{itemize}

\subsubsection{e}

\textbf{Read the following engineering urban legends and complete the descriptions of causes and effects using the correct form of the words and phrases in Exercise 13d. Sometimes more than one word or phrase is possible}:

\begin{itemize}

\item Apparently, the biggest challenge in space exploration was developing a pen for astronauts to use in orbit as ordinary ballpoint pens don't work in space, \textit{because of/ due to / owing to} the fact that there's no gravity. So \textit{because of} this problem, there were teams of researchers working for years, trying to find a solution. Eventually, someone came up with the idea of using a pencil.
\item When they designed the foundations of the library on the university campus, they forgot to allow for the weight of the books on the shelves, which \textit{caused} the building to start sinking. So \textit{consequently}  half of the floors have had to be left empty, without books, to keep the weight down.
\item Did you hear about that Olympic-sized swimming pool that was built? They got the length wrong \textit{because} of the tiles. They forgot to take into account the thickness, which \textit{resulted in} the pool measuring a few millimetres too short. So \textit{consequently}, it can't be used for swimming competitions.

\end{itemize} 



\section{UNIT 10: Pushing the boundaries}

\subsection{2}

\subsubsection{b}

\textbf{Match the words (1-6) from the discussion to the definitions (a-0)}:

\begin{itemize}

\item appropriate/suitable: \textit{good enough for the intended function};
\item consistent/reliable: \textit{makes the most of resources, isn't wasteful};
\item cost-effective/economical: \textit{performs a function well};
\item effective: \textit{the right solution for a particular situation};
\item efficient: \textit{doesn't break down, always performs in the same way};
\item sufficient/adequate: \textit{works quickly and well}.

\end{itemize}

\subsubsection{c}

\textbf{Make the following words negative by adding the prefixes in- or un-}:

\begin{itemize}

\item adequate: \textit{inadequate};
\item appropriate: \textit{inappropriate};
\item consistent: \textit{inconsistent};
\item economical: \textit{uneconomical};
\item effective: \textit{ineffective};
\item efficient: \textit{inefficient};
\item reliable: \textit{unreliable};
\item sufficient: \textit{insufficient};
\item suitable: \textit{unsuitable}.

\end{itemize}

\subsection{3}

\subsubsection{a}

\textbf{The following information is from the web site of Sigma Power, a firm that advises corporate and government clients on wind energy projects. Complete the text using the words in Exercise 2c}:

\begin{itemize}

\item The fact that wind turbines consume no fuel and waste very little energy is clearly a fundamental advantage. But just how \textit{efficient} are they? \underline{Key figures}
\item Clearly, wind turbines need to be located on relatively windy sites in order to function. From a meteorological standpoint, what kinds of \underline{geographical location} are the most \textit{suitable}?
\item Turbines are generally placed at the tops of tall towers, where wind speeds are higher, thus making them more \textit{effective}. What other \underline{positioning factors} influence performance?
\item Wind turbines rarely function continuously, due to the fact that wind speeds are \textit{inconsistent}. How significant is the impact of \underline{variable weather conditions} on power generating capacity?
\item Transmitting electricity over long distances is inherently \textit{inefficient}, due to power loss from overhead or underground power lines. \newline Find out more about \underline{the advantages of generating power locally};
\item The generating capacity of wind turbines is generally \textit{insufficient} for it to be relied upon 100\%. What percentage of total \underline{generating capacity} can wind turbines realistically provide?
\item Some early wind turbines were \textit{insufficient}, suffering breakdowns caused by inaxial stresses stemming from higher wind loads on the upper blade. However, this problem has been overcome on modern units. Learn more about the \underline{technical evolution of wind turbines}.

\end{itemize}

\subsection{4}

\subsubsection{d}

\textbf{Label the diagrams using the forces in Exercise 4c}:

\begin{itemize}

\item\textit{compression};
\item\textit{bending};
\item\textit{torsion/torque};
\item\textit{expansion};
\item\textit{pressure};
\item\textit{tension};
\item\textit{shear};
\item\textit{friction};
\item\textit{contraction};
\item\textit{centrifugal force}.

\end{itemize}

\subsubsection{e}

\textbf{Complete the following sentences from the talk using the forces in Exercise 4c. Listen again and check your answers}:

\begin{itemize}

\item So that downward force means the structure is in \textit{compression}, especially near the bottom;
\item a horizontal load, exerted by air \textit{pressure} against one side of the structure;
\item Because the structure is fixed at ground level, and free ot the top, that generates \textit{bending} forces;
\item when elements bend, you have opposing forces: \textit{compression} at one side, \textit{tension} at the other;
\item the wind effectively tries to slide the structure along the ground, and the foundations below the ground resist that. The result of that is \textit{shear} force;
\item the foundations need to rely on \textit{friction} with the ground to resist the pull-out force;
\item The action of the wind can also generate \textit{torsion/torque}. You get a twisting force;
\item When concrete absorbs heat from the sun, you get \textit{expansion}; as soon as the sun goes in, there's \textit{compression}.

\end{itemize}

\subsection{5}

\subsubsection{b}

\textbf{Read the extract from an article about transport in a popular science and technology magazine and answer the following questions}:

\begin{itemize}

\item What factors should be considered in the comparative analysis described? \textit{Speed, convenience, efficiency and environmental-friendliness};
\item What is the purpose of the comparative analysis? \textit{To find the best way of transporting people};
\item What suggestion is made about Europe? \textit{That high-speed electric trains are the most efficient solution}.

\end{itemize}

\subsection{5}

\subsubsection{c}

\textbf{Find words in the text in Exercise 5b to match to the following definitions. Which one of the words has a plural form?}

\begin{itemize}

\item standard by which you judge something: \textit{criterion};
\item fact or situation which influences the result of something: \textit{factor};
\item number, amount or situation which can change: \textit{variable}.

\end{itemize}

\subsection{6}

\subsubsection{d}

\textbf{Listen again and complete the following table about the modified TGV using the figures in the box}:

\begin{itemize}

\item Maximum speed: $+80\%$;
\item Train length (with coaches): $-50\%$;
\item Aerodynamic drag: $-15\%$;
\item Diameter of wheels: $+19\%$;
\item Motor power output: $+68\%$.

\end{itemize}

\subsubsection{e}

\textbf{Complete the following sentences from the talk by underlining the correct words}:

\begin{itemize}

\item The record speed exceeded the standard operating speed by a \textit{huge} margin;
\item The train was modified to a \textit{certain} extent;
\item the modified train was \textit{significantly} shorter;
\item changes were made to the bodywork, to make it \textit{slightly} more aerodynamic;
\item The wheels on the modified train were \textit{marginally} bigger;
\item the power of the electric motors was \textit{substantially} higher than the standard units;
\item standard high-speed trains can be made to go faster by a \textit{considerable} amount.

\end{itemize}

\subsubsection{f}

\textbf{Rewrite the following sentences to describe the modifications that were made to the TGV for the record attempt. Use the phrases in Exercise 6e to replace the words in bold}:

\begin{itemize}

\item The supply voltage in the catenary cables had to be increased \textbf{from $25,000$ to $3l,000$ volts}: \textit{The supply voltage in the catenary cables had to be increased by a considerable amount};
\item To limit oscillation, the tension of the catenary cables had to be increased \textbf{by $60\%$}: \textit{To limit oscillation, the tension of the catenary cables was substantially increased};
\item On some curves, the camber of the track had to be increased \textbf{by a few centimetres}: \textit{The camber of the track was increased marginally on some curves};
\item The $574.8\ km/h$ record beat the previous record, set in 1990, by $59.5\ km/h$: \textit{The previous record was beaten by a huge margin};
\item In perfect conditions the TGV could probably have gone faster by $5$ to $l0\ km/h$: \textit{In perfect conditions, the TCV could probably have gone slightly faster}.

\end{itemize}

\subsection{8}

\subsubsection{e}

\textbf{Complete the following groups of synonyms using the words in the box}:

\begin{itemize}

\item exposed to (a force): \textit{subjected to};
\item resist (a force): \textit{cope with}, \textit{withstand};
\item go beyond (a limit): \textit{exceed}, \textit{surpass};
\item suitable for (a use): \textit{intended for};
\item can: \textit{able to}, \textit{capable of};
\item can't: \textit{unable to}, \textit{incapable of}.

\end{itemize}

\subsubsection{f}

\textbf{Complete the following sentences about Sonic Wind using the correct form of the words in Exercise 8e}:

\begin{itemize}

\item The bolts fixing the camera to the sled had to \textit{cope with} high shear forces;
\item The sled's rockets were \textit{capable of} generating enormous thrust;
\item The pools at the end of the track were \textit{able to} stop the sled rapidly;
\item The skids on the sled had to \textit{withstand} high level of frictions;
\item At full speed, John Stapp was \textit{subjected to} several tonnes of air pressure;
\item The rear of the sled was \textit{unable to} resist the shock of deceleration, and broke off;
\item Doctors thought people were \textit{incapable of} surviving forces of $17\ Gs$ and above;
\item John Stapp \textit{exceeded} the $17\ G$ limit by a huge margin.

\end{itemize}