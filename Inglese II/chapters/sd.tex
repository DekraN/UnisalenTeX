% !TEX encoding = UTF-8
% !TEX TS-program = pdflatex
% !TEX root = ../eng.tex
% !TEX spellcheck = it-IT

%************************************************
\chapter{Specialized Discourse}
\label{cap:sd}
%************************************************\\

\section{SD: EARLY DEFINITIONS}

\begin{itemize}

\item\textbf{1920s-1930s – Prague School}

focus on the \textbf{functional style} of scientific/technical discourse, diverging from everyday texts at the level of 

\begin{itemize}

\item \underline{word morphology} (e.g., foreign words with original plural suffix; obsolete forms of verbs/adjectives); 
\item \underline{word formation} (e.g., use of classical prefixes, nominal premodifications, etc.).

\end{itemize}

\item\textbf{1940s-1960s – Halliday, McIntosh \& Strevens (1964)}

focus on the notion of \textbf{specialized register} as a language variety with specific morpho-syntactic, lexical and stylistic features diverging from common language in relation to:

\begin{itemize}

\item\underline{the topic of communication};
\item\underline{the ‘community of specialists’ using it}.

\end{itemize}

\end{itemize}


\section{THE ISSUE OF TERMINOLOGY}

Specialized discourse as:

\begin{itemize}

\item\textbf{Restricted language}:  i.e., standard messages using set phrases with only few established variants (cf. Dodson 1974; Wallace 1981) – inappropriate because specialized discourse exploits the language code in more creative ways;
\item\textbf{Special language}: using linguistic and non-linguistic conventions which may be absent from general language (e.g., language for maritime telecommunications) – inappropriate because specialized discourse uses conventions in more varied pragmatic ways;
\item\textbf{Microlanguage}: inappropriate for its reference to a microcosm lacking the expressive (lexical, morpho-syntactic and textual) richness of standard language;

\end{itemize}

\section{RECENT DEFINITIONS OF SPECIALIZED DISCOURSE}

\begin{itemize}

\item Gotti (2005): “Specialized discourse reflects the specialist use of language in contexts which are typical of a specialized community, stretching across the academic, the professional, the technical and the occupational areas of knowledge and practice.”;

\item Halliday (1978): specialized registers classified according to:

\begin{itemize}

\item\textbf{Mode} medium of communication;
\item\textbf{Field} topic of communication;
\item\textbf{Tenor} relationship between the participants in specialized interaction.

\end{itemize}

\item Turner (1980): use of jargon determining opacity in specialized discourse, depending on unfamiliar lexis and content for the unqualified participant:
 
\begin{itemize}

\item\textit{Patient (to Nurse)}: Good morning. I’m here to have my tonsils out.
\item\textit{Nurse (to GP)}: Doctor, there’s a patient here for a tonsillectomy.

\end{itemize}

\end{itemize}


\section{SD: MULTIDIMENSIONAL NATURE}

\textbf{Tenor-based distinctions}

\begin{itemize}

\item\textbf{Scientific exposition}:
specialized terminology with no explanation;

\item\textbf{Scientific instruction}:
specialist addressing non-specialists: explanation of specialized lexis for educational purposes;

\item\textbf{Scientific journalism}:
specialist providing technical information through everyday lexis drawing on the layman’s everyday experience;
 
\end{itemize}

\section{SD: GENERAL FEATURES}

\begin{itemize}

\item Hoffmann (1984): 11 pragmatic features of specialized discourse:

\begin{itemize}

\item exactitude, simplicity and clarity;
\item objectivity; 
\item abstractness;
\item generalization;
\item density of information;
\item brevity; 
\item emotional neutrality;
\item unambiguousness;
\item impersonality;
\item logical consistency; 
\item use of defined technical terms, symbols and figures. 

\end{itemize}

\item Inconsistency in Hoffmann’s criteria:

\begin{itemize}
\item clarity may conflict with simplicity;
\item unambiguousness may conflict with conciseness and abstractness.
\end{itemize}

\item Sager et al. (1980): 3 criteria ensuring maximum efficiency in specialized communication with minimum cognitive effort:

\begin{itemize}
\item economy;
\item precision;
\item appropriateness.
\end{itemize}

\end{itemize}

\section{SD: LEXICAL FEATURES}

\subsection{MONOREFERENTIALITY}

\begin{itemize}

\item not used to indicate that each term has only one referent, but that in a given context only one meaning is allowed;
\item semantic uniqueness of a word: prevalence of denotation; impossible substitution with a synonym, but only by its definition or paraphrase; 
\item it is limited to the disciplinary field in which a term is employed;
\item primarily due to the scientific community’s effort to avoid alternative terms for the same concept.

\end{itemize}

\underline{Results}: \textit{conciseness} and \textit{lack of ambiguity} among scientific community. 

\subsection{LACK OF EMOTION}

\begin{itemize}

\item words have connotation; specialized terms have denotation;
\item Lion: (zoology field) = specific feline species [denotation], (everyday use) = aggressiveness, majesty, etc. [connotation];
\item the tone of specialized texts may seem cold and artificial.

\end{itemize}

Hence: \underline{Neutral tone} of scientific discourse (informative purpose) vs.\newline emotive tone of \underline{argumentative discourse} (persuasive purpose).

\subsection{PRECISION}

\begin{itemize}

\item every term must point to its own concept;
\item it arose in response to the need for precision advocated in the 17th century;
\item see the case of the ‘Cambridge don’. 

\end{itemize}

\subsection{TRANSPARENCY}

\begin{itemize}

\item criterion valued by Lavoisier and Linnaeus, who respectively thought that scientific nomenclature has to convey facts and idea precisely;

\item every term’s meaning must be accessed immediately through its surface form:

\begin{itemize}

\item \underline{Medical discourse}: separate lexical components of a specialized term easily decodable to reconstruct the meaning of the whole world:

\item \textit{Gastroenterology: gastro} = stomach + \textit{entero} = intestine + \textit{logy} = study (i.e., \textit{the study of stomach and intestine}).. 

\end{itemize}

\end{itemize}

\subsection{CONCISENESS}

\begin{itemize}

\item criterion applied to word formation: concepts are expressed in the shortest possible form;
\item \textbf{merging of two lexemes into a single terms}:  \textit{informatique}, from \textit{information + automatique}; \textit{telematica}, from \textit{telecomunicazione + informatica}.
\item \textbf{reduction of the term itself}:  both \underline{internally} (e.g., \textit{contraception}, from \textit{contraconception}) and terminally (e.g., \textit{haemostat}, from \textit{haemostatic forceps});
\item \textbf{juxtapostion}: omitting prepositions and premodifiers in nominal groups containing two nouns: - \textit{capostazione}; \textit{estratto-conto};

\end{itemize}

\subsection{CONSERVATIVISM}

\begin{itemize}

\item types of specialized discourse (cf. Law) are estremely conservative, preserving linguistic traits and old formulae that have disappeared from everyday language, also avoiding monoreferentiality of Classical terms;

\end{itemize}

\underline{Purpose}: formulaic language used to ensure the action’s validity through reverence for tradition:

\begin{itemize}

\item\underline{antiquated legal lexis}:  whereas (= premesso che); whosoever (= chiunque); thereof (= al riguardo); fortwith (= immediatamente); henceforth (= d’ora innanzi);
\item  archaic word-forms: third-person singular –eth instead of –(e)s,          with the present simple of verbs – e.g., he witness\underline{eth}; she do\underline{th}.

\end{itemize}


\subsection{IMPRECISION}

\begin{itemize}

\item \underline{violation} of the principle of precision: cases of referential fuzziness of terms;
\item \underline{legal discourse}: against precision because of the use of adjectives often allowing subjective, arbitrary interpretations: 

\begin{itemize}

\item “The Tenant will pay for a proper proportion of the \textit{recurring} charges […]”;
\item “The Tenant will permit the Landlord at \textit{reasonable} hours in the daytime to enter the Property […]”.

\end{itemize}

\end{itemize}

\subsection{REDUNDANCY}

\begin{itemize}

\item \underline{violation} of the principle of conciseness: when the number of the lexemes used is far higher than necessare;
\item \underline{legal discourse}: redundant for its frequent violations of the conciseness principle:

\begin{itemize}

\item  two synonyms for the same concept: \textit{new} and \textit{novel}; \textit{false} and \textit{untrue}; \textit{terms and conditions};
\item redundant expressions in contracts: \textit{mutually agreed}; \textit{solemnly declared}; \textit{undertakes to employ};
\item redundant repetition of a concept through its negative opposite: \textit{within and not exceeding two months};
\item synonyms semantically distinct in earlier centuries:
\begin{itemize}
\item last will and testament: will = movables; testament = real estate;
\end{itemize}
\item witnesses’ oath in court:
\begin{itemize}
\item to tell the truth, the whole truth, and nothing than the truth. Not different truths, but reference to Aquina’s argument for not feeling compelled to tell “the whole truth”.
\end{itemize}
\end{itemize}

\end{itemize}

\subsection{METAPHOR}

\textbf{METAPHOR (to put “new senses into old words to remedy a gap in the vocabulary” — Black 1962)}:

\begin{itemize}

\item\underline{Advantages}:

\begin{itemize}
\item\textit{transparency} by semantic association with existing concepts;
\item\textit{conciseness} by immediate recall of existing information with no lengthy explanations;
\item\textit{effectiveness} of the presentation of new abstract concepts by means of concrete images and experiences from the physical world:

\begin{itemize}
\item\underline{Economic} discourse: elasticity of demand; economic depression;
\item\underline{Popular scientific discourse}: an atom is a tiny solar system; the brain is a computer.
\end{itemize}

\end{itemize}

\item\underline{Ambiguities}: 
\begin{itemize}
\item\textit{Parent company} = “a firm controlling another firm”, not “a firm that has established another firm” (which may be a ‘branch’).
\end{itemize}

\end{itemize}


\section{SD: SYNTACTIC FEATURES}

\begin{itemize}

\item The syntactic construction of a specialized discourse provides evidence of the specialists’ organization of their logical thought through language;
\item The specificity of the syntactic features is not qualitative, but quantitative;
\item Certain features may occur in general language, but their frequence in specialized discourse is higher;
\item Focus on:

\begin{itemize}
\item the main syntactic features most frequently occurring in specialized discourse;
\item the pragmatic reasons for which they occur.  
\end{itemize}

\end{itemize}

\subsection{OMISSION OF PHRASAL ELEMENTS}

\begin{itemize}

\item the principle of conciseness justifies the extremely compact syntactic structure of specialized discourse;
\item omission of some constituents of a sentence, especially in technical manuals focusing on instructions: 

\begin{itemize}

\item “Grasp each end of * patch. Stretch and roll * center of * patch into * eye of * needle. Remove * protective covering from both sides […]”; 
\item\underline{italian}: \textit{premere pulsante A; accusare ricevuta; presentare istanza; depositare denuncia}.

\end{itemize}

\end{itemize}

\subsection{HOW TO ACHIEVE CONCISENESS}

\begin{itemize}

\item Substitution and simplification of relative clauses:

\begin{itemize}

\item with adjectives obtained by means of affixation (prefixes and suffixes):

\begin{itemize}
\item workable metal = ‘a metal which can be worked’;
\item unproductive capital = ‘capital which does not produce goods’.
\end{itemize}

\item by omitting subject and auxiliary in passive voice:
\begin{itemize}
\item an instrument called a spectroscope = an instrument which is called a spectroscope;
\end{itemize}

\item by turning the verb into a past participle and using it as a premodifier:
\begin{itemize}
\item Compressed air can be used … = Air which is compressed can be used;
\end{itemize}
\item by placing the agent before the past participle and linking the two components by a hyphen:

\begin{itemize}
\item Computer-calculated result = A result which was calculated by a computer;
\end{itemize}

\item by joining the adverb that modifies the passive form to the past participle of the verb by a hyphen:

\begin{itemize}
\item “An incorrectly-designed bridge may have a short life” = A bridge which is designed incorrectly;
\end{itemize}

\end{itemize}

\item by using \textit{thus} and \textit{so} to:

\begin{itemize}

\item “the result thus/so obtained were inaccurate” = the results which were obtained in this way were inaccurate;
\item “[…] the air below the piston rises, thus causing the pressure to fall” = the air rises, and in this way it causes …

\end{itemize}

\item by using \textit{whereby} to avoid the extended adverbial phrase by means of which:

\begin{itemize}
\item cracking is the process whereby kerosene is extracted = cracking is the process by means of which kerosene is extracted 
\end{itemize}

\item transforming the verb of a relative clause into a present participle:

\begin{itemize}
\item Tungsten is a metal \textit{retaining} hardness = Tungsten is a metal which retains hardness
\end{itemize}

\item Sometimes, the strategies activated to achieve conciseness are somewhat ‘extreme’. Consider, by way of example:

\begin{itemize}
\item the disappearance of the subject and the auxiliary of the secondary clause and the verb itself:
\begin{itemize}
\item a pentagon is a figure which has five sides $\implies$ a pentagon is a five-sided figure;
\end{itemize}
\end{itemize}

\item the gradual simplification process leading to a noun specifying another noun:

\begin{itemize}
\item an engine which is driven by diesel oil $\implies$ an engine driven by diesel oil $\implies$ a diesel (oil)-driven engine $\implies$ a diesel engine;
\end{itemize}

\end{itemize}

\subsection{PREMODIFICATION}

\begin{itemize}

\item differently from Italian, English can also employ the right-to-left construction to shorten sentences and to make them denser. Premodification is a distinctive aspect of right-to-left construction, allowing:

\item\underline{nominal adjectivation} (the use of a noun to specify another noun):

\begin{itemize}
\item \textit{silicon chip} [material]; \textit{access program} [use]; \textit{control byte} [function];
\end{itemize}

\item\underline{merging two short nouns into a single term functioning as an adjective}, at first hyphenated, then a one-word compound:

\begin{itemize}
\item flow-chart $\implies$ flowchart; plug-board $\implies$ plugboard;
\end{itemize}

\item using attributive nouns instead of adjectives:

\begin{itemize}
\item gravity anomaly; energy-rich molecules.
\end{itemize}

\item ambiguity cases: specialists must rely on knowledge of both syntactic rules and their specialized discipline;
\item disambiguation may be achieved by the use of hyphen: 

\begin{itemize}
\item a small car-factory;
\item a small-car factory;
\item an L-shaped computer room.
\end{itemize}

\end{itemize}

\subsection{NOMINALIZATION - DEFINITION}

\begin{itemize}
\item the use of a noun instead of a verb to convey concepts concerning actions or processes;
\end{itemize}

\subsection{NOMINALIZATION - REALIZATION}

\begin{itemize}

\item nominal adjectivation and premodification: 

\begin{itemize}
\item a day and night weather observation station
\end{itemize}

\item verb nominalization weakens the verb, reducing it to the function of copula linking complex noun phrases. Non-copulative verbs are thus reduced into adjectival forms:

\begin{itemize}
\item oscillations are frequency-dependent = oscillations depend on frequency 
\end{itemize} 

\end{itemize}

\subsection{SENTENCE COMPLEXITY}

\begin{itemize}

\item simplification of sentence structure into: NOUN PHRASE + VERB + NOUN PHRASE;

\begin{itemize}

\item the complete development of the fracture model \textit{requires} an understanding of the bond-rupture reaction;
\item  the testing of machine by this method \textit{entails} sole loss of power.

\end{itemize}

\item  high number of non-finite forms (infinite, present/past participle), which may nonetheless complicate comprehension:

\begin{itemize}
\item  the proton is the opposite of electron, being a particle … = the proton is the opposite of electron, \textit{since it is} a particle …
\end{itemize}

use of \textit{thereby} + present participle to avoid clause subordination.
 
\end{itemize}
 
\subsection{SENTENCE LENGTH - LEGAL DISCOURSE}

\begin{itemize}

\item sentences are longer (and considerably longer in \underline{legal discourse}) to minimize ambiguity and misunderstandings by means of specification and clarifications; 
\item to specify with precision any reference to people, time and place:

“This Agreement, effective as of the first day of April 1987 between Dale Johnson Ryder Warren, an Association organized and existing under the laws of Switzerland (‘Grantor’), its successors and assigns, and DJRW Johnson Ryder Simpson \& C., its successors and assigns (‘Member Firm’).

\end{itemize}
 
\subsection{VERB TENSES}

\begin{itemize}

\item\underline{present/past participle} is used to achieve conciseness by simplifying relative clauses or omitting phrasal elements that are not considered important; yet, the general use of verb tenses is as follows:

\item\underline{present simple} is more used in expository scientific texts, whose pragmatic functions are: definition, description, stating general truths, postulating scientific laws, explaining standard procedures;

\item\underline{infinite} when relative clause is eliminated: 

\begin{itemize}
\item “The record to be located [= which is to be located] is searched in the file”; 
\end{itemize}

\item other verb tenses are found in argumentative texts, whose pragmatic functions may be several and the degree of generality attributed to the phenomenon described may be low.

\end{itemize}

\subsection{USE OF THE PASSIVE + DEPERSONALIZATION}

\begin{itemize}

\item Discourse is depersonalised by emphasizing the result of an action rather than its agent or doer:

\begin{itemize}
\item the experiment \textit{has been carried out} by the researcher … 
\end{itemize}

\item  whereas the passive voice may suggest a standardized procedure, the active voice identifies a special procedure developed by the author(s):

\begin{itemize}
\item in this paper \textit{we develop} the theory of … 
\end{itemize}

\item the active voice is prevalent in \underline{legal discourse}.

\end{itemize}


\section{SPECIALIZED DISCOURSE: TEXTUAL FEATURES}

Avoidance of standard textual norms, in favour of ‘deviant’ alternatives for pragmatic reasons. 

\subsection{ANAPHORIC REFERENCE} 

\begin{itemize}

\item The use of ‘pro-nouns’ to substitute nouns is common to increase cohesion, but this may be disregarded in specialized discourse;
\item In \underline{Legal discourse}, anaphoric reference is avoided in favour of lexical repetition, to achieve maximum clarity and avoid ambiguity:

\begin{itemize}
\item The Tenant will permit the \textit{Landlord} or the \textit{Landlord}’s agents … to enter his property 
\end{itemize}

\item Avoiding \textit{anaphoric reference} to achieve precision and clarity by a frequent reference to parts of the text itself by the adverbials \textit{hereto}, \textit{herein}, \textit{hereof}, \textit{thereto} (‘textual mapping’ – Bhatia 1987): 

\begin{itemize}
\item A copy of the agreement is attached \textit{hereto} as Appendix B (without Appendices A and B attached \textit{thereto} which are Appendix A \textit{hereto} and a form of this Agreement) and made a part \textit{hereof} is fully recited \textit{herein};
\end{itemize}

\item Using a lexical anaphora that does not repeat words but clarifies the illocutionary value (intentionality) of the word it refers to:

\begin{itemize}
\item The way to solve some problems connected with congestion in air traffic is to raise prices at peak times and lower them at others. This proposal is based on nothing more the principles of demand and supply.
\end{itemize}

\end{itemize}

\subsection{USE OF CONJUNCTIONS}

\begin{itemize}

\item Conjunctions add cohesion to texts, but also clarify the pragmatic purposes of the sentence that follows them: 

\begin{itemize}
\item \textit{but}; \textit{however}; \textit{on the other hand} introduce a sentence semantically opposed to the previous one; 
\item \textit{as}; \textit{since}; \textit{for}; \textit{because}; introduce a reason or explanation. 
\end{itemize}

\end{itemize}

\subsection{THEMATIC SEQUENCE}

\begin{itemize}
\item The thematic structure of specialized discourse (Halliday 1973; Halliday \& Martin 1993) focuses on a sequence of \textbf{THEME}(introducing the topic, or also ‘given’ information known to the addressee) and \textbf{RHEME} (‘new’ information not found in the preceding text).
\end{itemize}

\underline{\textbf{EXAMPLES OF THEME/RHEME STRUCTURE}}

“It is essential to \textit{keep} magnetic disks in \textit{fireproof} safes [Rheme]. \textit{This protection} [Theme] of original software, however, \textit{is not sufficient} [Rheme]. An \textit{additional security precaution} [Theme] consists in storing copies at different sites, away from the computer.  

\subsection{ARGUMENTATIVE PATTERN}

\begin{itemize}

\item Aim of argumentation: to convince readers that the author’s perspective is the right one;
\item ‘Compositional plan’ of the text (Weirlich 1976):
CLAIM based on DATA and supported by WARRANT yet: CLAIM undermined by REBUTTAL by a QUALIFIER (e.g., ‘presumably’) so: CLAIM strengthened by BACKING ITEM;
\item Overall pattern underlying argumentative texts (Toulmin 1958): 
 
DATA observation – PROBLEM identification – suggest of SOLUTION – argumentation supported by PROOF - CONCLUSION 

\item The author’s criticism to alternative claims is carried out by projecting his/her ‘authorial self’ in the text, by means of the first-person pronouns I, we: 

\begin{itemize}
\item \textit{I shall argue} that the postulates of the classical theory are applicable to a special case only and not to the general case.
\end{itemize}

\item by indirect expression of critical views by means of indefinite forms such as \textit{one}, \textit{someone}, or general nouns such as \textit{people}, \textit{the majority}, etc.:
 
\begin{itemize}
\item In the case of a change peculiar to a particular industry \textit{one} would expect the change in real wages to be in the same direction as the change in the moneywages;
\item \textit{Some people} still seem to accept that war is a risk of power politics. But \textit{the majority} would probably hold that war is a completely inappropriate means of politics
\end{itemize}

\item To make his/her persuasive function more effective, the author can appeal directly to the reader: 

\begin{itemize}
\item \textit{The reader} can notice that …
\end{itemize}

\item The author can choose not to mention the reader explicitly, preferring to use more impersonal sentences or a persuasive use of \textit{modals} such as \textit{must} or \textit{should}:

\begin{itemize}
\item These conclusions are intended, \textit{it must be remembered}, to apply to … 
\end{itemize}

\item The author can compel the reader to obey his/her argumentative instructions, choosing a neutral tone, as well as use of \textit{modals} indicating a logical conclusion based on evidence: 

\begin{itemize}
\item If employment increases, then the reward per unit in terms of wage-goods must, in general, decline and profits increase;
\end{itemize}

\end{itemize}

\subsection{THE EMOTIVE FORCE OF SPECIALIZED TEXTS}

\begin{itemize}

\item Objectivity of SD requires the removal of emotive content (Walton 1989);
\item \textbf{\textit{YET}}:
\item ‘persuasive rhetoric’ (Frye 1957) requires a reconsideration of the emotive element in argumentation;
\item \textbf{\textit{HENCE}}: use of figurative and emotive language also in SD to reinforce the persuasive aims of argumentation:

\begin{itemize}

\item use of adjectives such as: \textit{extreme}, \textit{extensive}, \textit{enormous}, \textit{disastrous}, \textit{violently}, \textit{prevailing}, etc. and attitudinal adverbials such as: \textit{obviously}, \textit{surely}, \textit{indeed}, \textit{of course}, \textit{frankly}, etc.; 

\item Day-to-day fluctuations in the profits of existing investments tend to have an \textit{extensive}, and even an \textit{absurd}, influence on the market. 

\end{itemize}

\end{itemize}


\section{THE LEXIS OF COMPUTER SCIENCE}

\begin{itemize}

\item\textbf{Main features of the lexis of computer science}

\item\underline{Information Science}: new specialized field with specific concepts and tools, developing its own terminology. Its lexis is often derived from: 

\begin{itemize}
\item other disciplines;
\item general English.
\end{itemize}

\item Focus on \underline{technical words} used by specialists to communicate with other specialists as regards computer applications.

\end{itemize}

\subsection{COMPUTER SCIENCE: SPECIALIZATION AND BORROWING} 

\subsubsection{WORDS BORROWED FROM THE GENERAL LANGUAGE}

\begin{itemize}

\item

\begin{itemize}
\item hardware;
\item chat group;
\item program;
\item disk.
\end{itemize}

\item Use of American English (AE) instead of British English (BE) because of the US computer industry:

\begin{itemize}
\item program (AE) vs. programme (BE);
\item disk (AE) vs. disc (BE). 
\end{itemize}

\end{itemize}

\subsubsection{METAPHORS}

reformulation of new concepts to create analogy (which, however, is often misleading):

\begin{itemize}

\item \textit{memory}; \textit{bus};
\item \textit{gate}; \textit{store};
\item \textit{menu}; \textit{domain};
\item \textit{mouse} [singular $\implies$ \textit{mouses} – regular plural $\implies$ \textit{mice} – irregular plural in general English];
\item spamming [= brand of canned meat].  

\end{itemize}

\subsubsection{CHANGE OF GRAMMATICAL CATEGORY}

\begin{itemize}

\item \textit{e-mail / emails} $\implies$ \textit{e-mail} (verb) [but \textit{mail} is uncountable in English];
\item \textit{format} [noun and verb $\implies$ only noun in general English];
\item \textit{peripheral} [noun and adjective $\implies$ only adjective in general English]. 

\end{itemize}

\subsubsection{USE OF COMPRESSION TECHNIQUE}

\begin{itemize}

\item \textit{alphametric} from \textit{alphanumeric};
\item \textit{digitize} from \textit{digitalize};
\item \textit{optronics} from \textit{optoelectronics}.

\end{itemize}

\subsection{COMPUTER SCIENCE: NEOLOGY}

\begin{itemize}

\item \underline{Creation of new words} when they cannot be borrowed from general English, or specialized languages, or foreign languages;
\item \underline{Derivation processes by}:

\begin{itemize}
\item Association:
\begin{itemize}
\item \textit{byte} = a blend of \textit{bit} (‘morsel’) and \textit{bite} (‘chew’), but also acronym of \textit{Binary digIT Eight};
\end{itemize}
\item Affixation (suffixes/prefixes):
\begin{itemize}
\item \textit{Autocode}; \textit{kilobyte}; \textit{megabit}; \textit{postprocessing}; \textit{nonformatted}; \textit{multiaddress};
\end{itemize}
\item Analogy (new words modelled on already existing lexeme):
\begin{itemize}
\item \textit{software} from \textit{hardware};
\end{itemize}
\end{itemize}

\item\underline{Derivation process by}:

\begin{itemize}

\item Simile: (new expressions referring to aspect or category of an item):

\begin{itemize}
\item \textit{bridge connector}; \textit{banana plug}; \textit{star connection}; …
\end{itemize}

\item Compounding economy (new expressions as short and concise as possible):

\begin{itemize}
\item \textit{computer programmer} from \textit{programmer of computers};
\end{itemize}

\item Material specification:
\begin{itemize}
\item \textit{ferrite core}; \textit{silicon chip};
\end{itemize}

\item Use specification:
\begin{itemize}
\item \textit{access arm}; \textit{control byte}; \textit{load program}; … 
\end{itemize}

\end{itemize}

\end{itemize}

\subsection{COMPUTER SCIENCE: ACRONYMS AND ABBREVIATION}

\begin{itemize}

\item \underline{Acronyms to make group of words as concise as possible}:
\begin{itemize}

\item \textit{ASCII = American Standard Code for Information Interchange};
\item \textit{RAM = Random Access Memory};
\item \textit{ROM = Read Only Memory}.
\end{itemize}

\item \underline{Acronyms suggesting an implied meaning}:

\begin{itemize}

\item \textit{BASIC = Beginners’ All-purpose Symbolic Instruction Code};
\item \textit{EDIT = Error Deletion by Iterative Transmission}.
\end{itemize}

\item \underline{Abbreviations}:

\begin{itemize}

\item \textit{CPU = Central Processing Unit};
\item \textit{ALU = Arithmetic and Logical Unit};
\item \textit{EDP = Electronic Data Processing}.
\end{itemize}

\end{itemize}

\subsection{COMPUTER SCIENCE: RECENT DEVELOPMENT}

\begin{itemize}

\item\underline{Suffixation}

\begin{itemize}

\item\textbf{-er} = referred to person performing a certain action:

\begin{itemize}
\item \textit{emailer} (= email sender); \textit{nternetter} (= Internet user); \textit{lurker} (bulletin subscriber who reads but doesn’t contribute to it); …
\end{itemize}

\item \textbf{-ing} = deriving nouns from verbs and referred to processes:

\begin{itemize}
\item \textit{cyberizing} (= causing someone’s interest in computer use); \textit{netwriting} (= writing on the Internet); …
\end{itemize}

\item \textbf{-ie} = referred to newcomers to the field of computer science:

\begin{itemize}
\item \textit{nettie} and \textit{newbie} (= inexperienced Internet user). 
\end{itemize}

\end{itemize}

\item\underline{Prefixation}

\begin{itemize}

\item \textbf{re-} = denoting repetition of an operation:

\begin{itemize}
\item \textit{remailer} (= network-connected computer that takes email, sends it on to a destination, and places a veil between sender and receiver);
\end{itemize}

\item \textbf{cyber-} = [from ‘cybernetics’] creating compound neologisms referred to computer usage:

\begin{itemize}

\item \textit{cyberboard}; \textit{cyberchat}; \textit{cybercrime}; \textit{cybersex};
\item \textit{cyber culture}; \textit{cyber science}; \textit{cyber world}.
\end{itemize}

\item \textbf{info-} = [from ‘information’] creating neologisms:

\begin{itemize}
\item \textit{Infomania}; \textit{infomercial}; \textit{infotainment}.
\end{itemize}

\end{itemize}

\item\underline{Analogic derivation}

\begin{itemize}

\item \textit{offline / online reader} (= reading email messages with the Internet ‘off’ or ‘on’);
\item \textit{Internaut} (from ‘austronaut’ = one who explores the Internet). 

\end{itemize}

\end{itemize}

\section{POPULARIZATION}

\begin{itemize}

\item

\begin{defn}{\textbf{Popularization}}

The conveyance of specialist knowledge for education or information purposes for an audience of nonspecialists;

\end{defn}

\item Difference between \underline{pedagogic texts} (a) and \textit{popularizations}:

\begin{itemize}

\item (a) = systematically providing students with conceptual and terminological resources suited to the subject content (“secondary culture”—Widdowson 1979), such as undergraduate textbooks, instruction manuals;

\item (b) = providing a wide reading public with specialized topics through everyday language and experience (thus extending the reader’s knowledge). 

\end{itemize}

\end{itemize}

\subsection{POPULARIZATION AND TRANSLATION}

\begin{itemize}

\item Popularization is close to translation in that it involves the transformation of a \underline{source text} (the specialized text) into a \underline{derived text} (the popular text);

\item Popularization as ‘redrafting’ that does not alter the disciplinary content but its language to suit a new target audience (‘intralinguistic translation’); 

\item Its language presents a wide use of metaphors and similes linking specialized content with the public’s general knowledge.

\end{itemize}

\subsection{POPULARIZATION: FEATURES}

\begin{itemize}

\item Absence of argumentative patterns stressing authorial innovations:
\begin{itemize}
\item ‘I have called’; ‘I argue that’; ‘I mean by this’;
\end{itemize}

\item Prevalence of the informative function;
\item Use of definitions;
\item Schematic distance between Senders and Recipients;
\item Novel forms of popularization, resorting to humorous discourse. 
\end{itemize}

\subsection{POPULARIZATION: DEFINITIONS}

\begin{itemize}

\item Absence of the first-person author or the third-person originator of definitions because the popularising purpose is ‘informative’, not ‘innovative’;

\item Preference for impersonal or passive forms:

\begin{itemize}
\item A ‘doughnut complex’ is a term that has been applied to describe […] a decaying central city and a ring of prosperous and growing suburban region”;
\end{itemize}

\item Reference to an entire category or profession originating a definition:

\begin{itemize}
\item “relict being the name scientists give to an animal or plant that was once widespread but is now confined to a small area”.
\end{itemize}

\item Juxtaposition $\implies$ specialized term + periphrasis, separated by a comma, dash, or parenthesis [1]; or periphrasis + specialized term [2]; 

\begin{itemize}

\item [1] “More than 99\% of atmospheric water is in the troposphere, the turbulent, weather-producing zone below about 40,000 feet” – \textbf{DEDUCTIVE PROCESS};
\item [2] “After 25 years of repeated review, an injectable synthetic hormone, Depo-Provera, was approved by the Food and Drug Administration last year”. – \textbf{INDUCTIVE PROCESS}.   

\end{itemize}

\item Term + Definition joined by expressions as ‘called’, ‘known as’, ‘that is’, meaning’, etc.:

\begin{itemize}
\item “The M.I.T. experimenters had to work with atoms in special states known as Rydberg states;
\end{itemize}

\item Use of ‘technically’ to signal the divergence of specialized definition from nonspecialists’ definition: 

\begin{itemize}
\item “disparities in the timing of signals reaching the two ears — technically called interaural time differences”.
\end{itemize}

\item Use of the disjunctive conjunction ‘or’:

\begin{itemize}
\item “Most polymers are nothing more that identical molecular units, or monometers”;
\end{itemize}

\item Semantic approximation through periphrasis introduced by ‘a little’, ‘like’, ‘a sort of’, etc:

\begin{itemize}
\item “The brain is \textit{a sort of} computer”. 
\end{itemize}

\item Figurative, everyday language taken introduced by ‘in other words’, ‘so-called’,  etc.:

\begin{itemize}
\item “female birds represent the greatest source of stress for the males. In other words, the stress has a \textit{so-called} femme fatale effect on the male”.
\end{itemize}

\item Metaphors \& similes; inverted commas (approximation):

\begin{itemize}
\item “the reactor core, by absorbing neutrons, could be used to ‘\textit{breed}’ new plutonium for reuse in the core”.
\end{itemize}

\item Etymological or explicative remarks (informative role):

\begin{itemize}
\item “pharmacogenetics, \textit{the study of genetics differences in response to drugs} […]”;
\end{itemize} 

\item Critical views often introduced by ‘perhaps’:

\begin{itemize}
\item “\textit{Perhaps} more novelty lies in what we call empty space”. 
\end{itemize}

\end{itemize}

\subsection{POPULARIZATION: SCHEMATIC DISTANCE} 

\begin{itemize}

\item  Popularization can be generally analysed according to (cf. Whitley 1985):

\begin{itemize}
\item the audiences for scientific knowledge;
\item the producers of knowledge;
\item the knowledge itself and its transformation;
\item the effects upon the production and validation of new knowledge.
\end{itemize}

\item Popularization represents the socio-cognitive contrast between the producers — who consider themselves part of an ‘élite group’ — and receivers;

\item Recall the distinction between scientific exposition and scientific journalism. 

\end{itemize}

\subsection{POPULARIZATION: NOVEL FORMS}

\begin{itemize}

\item The ‘expansion and differentiation’ (Whitley 1985) of sciences has broadened the scope of popularization and provided novel means for its realization;
\item For example, popularization may be achieved by integrating its peculiar linguistic knowledge and the strategies of humorous discourse:

\begin{itemize}
\item \textit{The Big Bang Theory};
\item \textit{1000 Ways to Die};
\item \textit{Curious \& Unusual Deaths}.
\end{itemize}

\item These novel forms of popularization integrate the informative and entertaining features. 

\end{itemize}

\section{MULTIMODAL POPULARIZATION}

\begin{itemize}

\item

\begin{defn}{\textbf{Multimodal Popularization}}

Types of popularization where the conventional linguistic and novel audiovisual strategies of reformulation of specialized knowledge interact.

\end{defn}

\item\underline{Aim}:

to increase the receivers’ knowledge, to inform by means of general language, not aiming at developing their secondary culture;
 
\item\underline{Sharing popularization main features}:

\begin{itemize}
\item lack of knowledge advancement;
\item focus on receivers.
\end{itemize}

\item\underline{Translation strategies}:

\begin{itemize}
\item to develop equivalent target scripts;
\item to respect the different illocutionary dimensions in recent, hybrid documentaries.
\end{itemize}

\item\underline{Lexical and syntactic features}:

\begin{itemize}
\item adoption of \textbf{periphrasis}; juxtaposition of \textbf{definitions}, mainly by means of the deductive logical process;
\end{itemize}

\item\underline{Multimodal construction}:

\begin{itemize}

\item ‘demand images’ when specialists take the floor to mark highstatus participants and communicate objectivity;
\item ‘offer images’ to represent everyday situations (docufiction).
\end{itemize}

\item\underline{Humorous discourse}:

\begin{itemize}

\item integration of \textbf{cognitive}, \textbf{superiority}, \textbf{derogatory} and \textbf{safe/arousal} strategies to aim to a larger group of non-experts;
\item depending on the construction of humorous discourses, receivers may be accomplices and targets, or only accomplices of jokes. 
\end{itemize}

\end{itemize}